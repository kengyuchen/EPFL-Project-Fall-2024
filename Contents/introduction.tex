%%%%%%%%%%%%%%%%%%%%

% Project Name: Semester Project Fall 2024 for EPFL
% File: introduction.tex
% Author: Keng-Yu Chen

%%%%%%%%%%%%%%%%%%%%

Biometric authentication offers an error-tolerant approach to user verification. Despite its convenience, unlike traditional authentication methods, servers have to verify users' identities by comparing the similarity of enrolled and probed data instead of their equivalence. An authentication method based on comparing hashes of two templates thus fails. Additionally, unlike a user-defined password, biometrics reveal sensitive personal information and cannot be changed, raising significant privacy concerns. Furthermore, the inherent nature of biometrics data can introduce a non-negligible false positive rate. These issues make designing a biometric authentication scheme and analyzing its security challenging and highlight the importance of a rigorous study in this domain.

Previous works have demonstrated several potential cryptographic primitives that can be utilized to instantiate a biometric authentication scheme, such as function-hiding inner-product functional encryption (fh-IPFE) \cite{cryptoeprint:2016/440, cryptoeprint:2018/1214,10.1007/978-3-030-90567-5_33, 10.1145/3488932.3497754, cryptoeprint:2023/481}, homomorphic encryption \cite{10.1007/978-3-642-40588-4_5, pradel2021privacypreservingbiometricmatchingusing}, fuzzy extractor \cite{10.1145/1030083.1030096, 7980010}, oblivious transfer \cite{cryptoeprint:2012/586}, relational hash \cite{cryptoeprint:2014/394}, etc. Some of them provide non-interactive protocols in the sense that only the clients transmit enrollment and probe messages to the server before the server decides the authentication results. On the other hand, an interactive protocol allows the server to send hints or challenges to the clients during the authentication process.

In this project, we provide a general framework for analyzing a non-interactive biometric authentication scheme.
We first formally define a biometric authentication scheme in Section \ref{sec:formalization}, which can be split into two layers: the biometric layer and the cryptographic layer.
The biometric layer accounts for collecting biometric data from users, comparing the closeness of enrolled and probed biometric templates, and deciding the authentication result.
The cryptographic layer, on the other hand, is to protect users' privacy and strengthen the security of the scheme.
Next in Section \ref{sec:security_game}, we describe two security games to model two security notions we consider relevant to a biometric authentication scheme: the unforgeability (UF) game and indistinguishability (IND) game. The UF game models an adversary's ability to impersonate the user by offering a (possibly invalid) probe message that can result in a succesful authentication. In the UF game, we consider several options for the adversary to add flexibility to our security model. The IND game evaluates the server's knowledge of users' biometrics, where we model the adversary's ability to recognize the biometrics through enrollment and probe messages.

In Section \ref{sec:security_analysis:fh-IPFE} and \ref{sec:security_analysis:rh}, we provide analyses of instantiations of biometric authentication schemes using two primitives: function-hiding inner-product functional encryption and relational hash, repsectively. We first introduce the security of these primitives, and then we provide reductions from the UF and IND security of their instantiated authentication schemes to the security of the primitives. Our results demonstrate necessary and sufficient conditions and provide transformations for these instantiations to achieve our desired security.

