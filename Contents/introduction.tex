%%%%%%%%%%%%%%%%%%%%

% Project Name: Semester Project Fall 2024 for EPFL
% File: introduction.tex
% Author: Keng-Yu Chen

%%%%%%%%%%%%%%%%%%%%

Biometric authentication offers an error-tolerant approach to user verification. Despite its convenience, unlike traditional authentication methods, servers have to verify users' identities by comparing the similarity of enrolled and probed data instead of their equivalence. An authentication method based on comparing hashes of two templates thus fails. Additionally, unlike a user-defined password, biometrics reveal sensitive personal information and cannot be changed, raising significant privacy concerns. Furthermore, the inherent nature of biometrics data can introduce a false positive rate that is not negligible. These issues make designing a biometric authentication scheme and analyzing its security challenging and highlight the importance of a rigorous study in this domain.

\paragraph{Previous Work}
Previous works have demonstrated several potential cryptographic primitives that can be utilized to instantiate a biometric authentication scheme, such as function-hiding inner product functional encryption (fh-IPFE) \cite{cryptoeprint:2016/440, cryptoeprint:2018/1214, 10.1007/978-3-030-90567-5_33, 10.1145/3488932.3497754, cryptoeprint:2023/481}, homomorphic encryption \cite{10.1007/978-3-642-40588-4_5, pradel2021privacypreservingbiometricmatchingusing}, fuzzy extractor \cite{10.1145/1030083.1030096, 7980010}, oblivious transfer \cite{cryptoeprint:2012/586}, relational hash \cite{cryptoeprint:2014/394}, etc. Some of them provide non-interactive protocols in the sense that only the clients transmit enrollment and probe messages to the server before the server decides the authentication results. On the other hand, an interactive protocol allows the server to send hints or challenges to the clients during the authentication process.

\paragraph{Explanation}

[A brief explanation of this work]

\paragraph{Contribution}
In this project, we present the following contributions:
\begin{itemize}
	\item We provide a new general framework for analyzing a non-interactive biometric authentication scheme. Our framework formalizes a biometric authentication scheme by splitting it into two layers: the biometric layer and the cryptographic layer. The biometric layer accounts for collecting biometric data from users, comparing the closeness of enrolled and probed biometric templates, and deciding the authentication result. The cryptographic layer, on the other hand, is to protect users' privacy and strengthen the security of the scheme.
	\item We list two security games to model two security notions that we consider relevant to a biometric authentication scheme: the unforgeability (UF) game and the indistinguishability (IND) game. The UF game models an adversary's ability to impersonate the user by offering a (possibly invalid) probe message that can result in a successful authentication, which is similar to the unforgeability notion in \cite{cryptoeprint:2014/394} and the malicious adversary in \cite{cryptoeprint:2023/481}. However, we consider several options for the adversary to add more flexibility to our security model.

	The IND game evaluates the server's knowledge of users' biometrics, where we model the adversary's ability to recognize the biometrics. Previous works \cite{cryptoeprint:2014/394, cryptoeprint:2018/1214, 10.1007/978-3-030-90567-5_33, cryptoeprint:2023/481} consider a similar security notion by considering an adversary who has enrollment and probe messages. Compared to them, our model also provides the adversary with oracles to users' biometric templates to take biometric distributions into consideration and increase practicality.
	
	\item We analyze the UF and IND security of existing instantiations of biometric authentication schemes from previous works. Our results demonstrate necessary and sufficient conditions and provide transformations for these instantiations to achieve our desired security.
\end{itemize}

\paragraph{Structure of the Project}
In Section \ref{sec:formalization}, we formally define a biometric authentication scheme, including the biometric layer and cryptographic layer. 
In Section \ref{sec:security_game}, we introduce the unforgeability (UF) game and the indistinguishability (IND) game. 
In Section \ref{sec:security_analysis:fh-IPFE} and \ref{sec:security_analysis:rh}, we recall two instantiations using function-hiding inner-product functional encryption and relational hash, repsectively, and provide analyses of the UF and IND security of them.

\paragraph{Notation}
In this project, we assume

\begin{itemize}
	
	\item $\lambda$ is the security parameter.

	\item $[m]$ denotes the set of integers $\{1, 2, \cdots, m\}$.

	\item $\Z_q$ is the finite field modulo a prime number $q$.

	\item A function $f(n)$ is called \emph{negligible} iff for any integer $c$, $f(n) < \frac{1}{n^c}$ for all sufficiently large $n$. We write it as $f(n) = \negl$, and we may also use $\negl$ to represent an arbitrary negligible function.
	
	\item $\poly$ is the class of polynomial funcions. We may also use $\poly$ to represent an arbitrary polynomial function.
	
	\item We write sampling a value $r$ from a distribution $\mathcal{D}$ as $r \getsdollar \mathcal{D}$. If $S$ is a finite set, then $r \getsdollar S$ means sampling $r$ uniformly from $S$.

	\item The distribution $\mathcal{D}^t$ denotes $t$ identical and independent distributions of $\mathcal{D}$.

	\item A PPT algorithm denotes a probabilistic polynomial time algorithm. Unless otherwise specified, all algorithms run in PPT.

\end{itemize}


