%%%%%%%%%%%%%%%%%%%%

% Project Name: Semester Project Fall 2024 for EPFL
% File: preliminaries.tex
% Author: Keng-Yu Chen

%%%%%%%%%%%%%%%%%%%%

%We introduce three types of inner product functional encryption schemes: function hiding functional encryption, two-input functional encryption, and two-client functional encryption. We will instantiate our biometric authentication scheme using these primitives.

We introduce two primitives to instantiate a biometric authentication scheme: function-hiding inner product functional encryption and relatioanl hash.

%\begin{definition}[Two-Input Inner Product Functional Encryption (adapted from \cite{cryptoeprint:2022/441})]
%\label{def:2i-IPFE}
	%A \emph{two-input inner product functional encryption} (2i-IPFE) scheme $\sf FE$ for a field $\mathbb{F}$ and input length $k$ is composed of PPT algorithms $\textsf{FE.Setup}$, $\textsf{FE.KeyGen}$, $\textsf{FE.Enc}$, and $\textsf{FE.Dec}$:

	%\begin{itemize}
	
		%\item $\textsf{FE.Setup}(1^\lambda) \to \textsf{sk}, \textsf{ek}_1, \textsf{ek}_2$: It outputs a secret key $\textsf{sk}$ and two encryption keys $\textsf{ek}_1, \textsf{ek}_2$.
	
		%\item $\textsf{FE.KeyGen}(\textsf{sk}, \mathbf{A}) \to \textsf{dk}_\mathbf{A}$: It generates the functional decryption key $\textsf{dk}_\mathbf{A}$ for a diagonal matrix $\mathbf{A} \in \mathbb{F}^{k \times k}$,  
	
		%\item $\textsf{FE.Enc}(\textsf{ek}_i, \mathbf{x}) \to \mathbf{c_x}$: Given an encryption key, either $\textsf{ek}_1$ or $\textsf{ek}_2$, it encrypts the input vector $\mathbf{x} \in \mathbb{F}^k$ to the ciphertext $\mathbf{c_x}$. 
	
		%\item $\textsf{FE.Dec}(\textsf{dk}_\mathbf{A}, \mathbf{c_x}, \mathbf{c_y}) \to z$: It outputs a value $z \in \mathbb{F}$.
	
	%\end{itemize}
	
	%\noindent \textbf{Correctness}: A 2i-IPFE scheme \textsf{FE} is \emph{correct} if $\forall (\textsf{sk}, \textsf{ek}_1, \textsf{ek}_2) \gets \textsf{FE.Setup}(1^\lambda), \mathbf{A} \in \mathbb{F}^{k \times k}$, and $\mathbf{x}, \mathbf{y} \in \mathbb{F}^k$, we have

	%\[
		%\textsf{FE.Dec}(\textsf{FE.KeyGen}(\textsf{sk},  \mathbf{A}), \textsf{FE.Enc}(\textsf{ek}_1, \mathbf{x}), \textsf{FE.Enc}(\textsf{ek}_2, \mathbf{y}) ) = \mathbf{x} \mathbf{A} \mathbf{y}^T \in \mathbb{F}.
	%\]

%\end{definition}

%Instantiation using a 2i-IPFE is given in Section \ref{sec:2i-IPFE-instantiation}.

%\begin{definition}[Two-Client Inner Product Functional Encryption (adapted from \cite{cryptoeprint:2022/441})]
%\label{def:2c-IPFE}
	%A \emph{two-client inner product functional encryption} (2c-IPFE) scheme $\sf FE$ for a field $\mathbb{F}$ and input length $k$ is composed of PPT algorithms $\textsf{FE.Setup}$, $\textsf{FE.KeyGen}$, $\textsf{FE.Enc}$, and $\textsf{FE.Dec}$:

	%\begin{itemize}
	
		%\item $\textsf{FE.Setup}(1^\lambda) \to \textsf{sk}, \textsf{ek}_1, \textsf{ek}_2$: It outputs a secret key $\textsf{sk}$ and two encryption keys $\textsf{ek}_1, \textsf{ek}_2$.
	
		%\item $\textsf{FE.KeyGen}(\textsf{sk}, \mathbf{A}) \to \textsf{dk}_\mathbf{A}$: It generates the functional decryption key $\textsf{dk}_\mathbf{A}$ for a diagonal matrix $\mathbf{A} \in \mathbb{F}^{k \times k}$,  
	
		%\item $\textsf{FE.Enc}(\ell, \textsf{ek}_i, \mathbf{x}) \to \mathbf{c_x}$: Given a label $\ell$ and an encryption key, either $\textsf{ek}_1$ or $\textsf{ek}_2$, it encrypts the input vector $\mathbf{x} \in \mathbb{F}^k$ to the ciphertext $\mathbf{c_x}$. 
	
		%\item $\textsf{FE.Dec}(\textsf{dk}_\mathbf{A}, \mathbf{c_x}, \mathbf{c_y}) \to z$: It outputs a value $z \in \mathbb{F}$.
	
	%\end{itemize}
	
	%\noindent \textbf{Correctness}: A 2c-IPFE scheme \textsf{FE} is \emph{correct} if $\forall (\textsf{sk}, \textsf{ek}_1, \textsf{ek}_2) \gets \textsf{FE.Setup}(1^\lambda), \mathbf{A} \in \mathbb{F}^{k \times k}$, label $\ell$, and $ \mathbf{x}, \mathbf{y} \in \mathbb{F}^k$, we have
	%\[
		%\textsf{FE.Dec}(\textsf{FE.KeyGen}(\textsf{sk},  \mathbf{A}), \textsf{FE.Enc}(\ell, \textsf{ek}_1, \mathbf{x}), \textsf{FE.Enc}(\ell, \textsf{ek}_2, \mathbf{y}) ) = \mathbf{x} \mathbf{A} \mathbf{y}^T \in \mathbb{F}.
	%\]

%\end{definition}

%Instantiation using a 2c-IPFE is given in Section \ref{sec:2c-IPFE-instantiation}.



