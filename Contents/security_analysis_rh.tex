%%%%%%%%%%%%%%%%%%%%

% Project Name: Semester Project Fall 2024 for EPFL
% File: security_analysis_rh.tex
% Author: Keng-Yu Chen

%%%%%%%%%%%%%%%%%%%%

Let $\Pi$ be an authentication scheme instantiated by a relational hash scheme \textsf{RH} as in Section \ref{sec:rh-instantiation}. We discuss the UF and IND security of $\Pi$ in this section. Note that in this instantiation, $\textsf{esk}, \textsf{psk}, \textsf{csk}$ are all public hash keys $\textsf{pk}$ of \textsf{FE} and assumed to be given to all adversaries.

Given a relational scheme \textsf{RH} for a relation $R \subseteq X \times Y$, we first define the unforgeability \cite{cryptoeprint:2014/394} of \textsf{RH}.

\begin{definition}[Unforgeability]

A relational hash scheme \textsf{RH} is called \emph{unforgeable} for the distribution $\mathcal{X}$ if for any adversary $\mathcal{A}$, the following probability is negligible.
\[
	\Pr \left [
		\begin{aligned} 
			 &\; \mathbf{x} \getsdollar \mathcal{X} \\
			 &\; \textsf{pk} \gets \textsf{RH.KeyGen}(1^\lambda) \\
			 &\; \mathbf{h_x} \gets \textsf{RH.Hash}_1(\textsf{pk}, \mathbf{x}) \\
			 &\; \mathbf{\tilde{z}} \gets \mathcal{A}(\textsf{pk}, \mathbf{h_x})
		\end{aligned} :
		\textsf{RH.Verify}(\textsf{pk}, \mathbf{h_x}, \mathbf{\tilde{z}}) = 1
		\right ] = \negl.
\]

\end{definition}

%Note that in \cite{cryptoeprint:2014/394}, the authors proposed an unforgeable relational scheme for the Hamming distance proximity relation $R^\tau$ for any $\tau < k$.

%-------------------

\subsection{UF Security of $\Pi$}
\label{sec:security_analysis:rh:uf}

We first consider $\textsf{option}$ that includes $\mathbf{c_x}$.


\begin{theorem}
\label{thm:rh:uf-uf-cx}

Let $\textsf{option} = \{\mathbf{c_x}, \textsf{esk}, \textsf{psk}, \textsf{csk}\}$. If \textsf{RH} is unforgeable for the distribution
\[
	\mathcal{X} = \{ \mathcal{B} \getsdollar \mathbb{B}: \mathbf{b} \gets {\sf getEnroll}^{\mathcal{O}_\mathcal{B}}() \mid \mathbb{B} \},
\]
then $\Pi$ is $\textsf{option}$-UF. 

\end{theorem}

In \cite{cryptoeprint:2014/394}, the authors construct an $\textsf{RH}$ that is unforgeable for the uniform distribution over $\{0, 1\}^k$, under the hardness of some computational problems. Note that we need to provide knowledge of $\mathbb{B}$ in the distribution $\mathcal{X}$.

\begin{proof}

Recall that the distribution of $\mathbf{c_x}$ in the \textsf{UF} game of the instantiation of Section \ref{sec:rh-instantiation} is
\[
	\left \{
		\begin{aligned} 
			 & \mathcal{B} \getsdollar \mathbb{B} \\
			 & \textsf{pk} \gets \textsf{RH.KeyGen}(1^\lambda) \\
			 & \mathbf{x} = \mathbf{b} \gets \textsf{getEnroll}^{\mathcal{O}_\mathcal{B}}() 
		\end{aligned} :
		\mathbf{c_x} \gets \textsf{RH.Hash}_1(\textsf{pk}, \mathbf{x})
	\right \}
\]
Also recall that $\textsf{Verify}(\textsf{Compare}(\textsf{csk}, \mathbf{c_x}, \mathbf{\tilde{z}} )) = \textsf{RH.Verify}(\textsf{pk}, \mathbf{c_x}, \mathbf{\tilde{z}} )$.
The \textsf{option}-UF security is thus guaranteed by the unforgeability of \textsf{RH}.

\end{proof}

\paragraph{Remark}
As we mentioned in Section \ref{sec:uf_game}, an adversary with \textsf{psk} can enjoy a winning rate of the false positive rate \textsf{FP} of $\mathbb{B}$. Theorem \ref{thm:rh:uf-uf-cx} thus implies that if $\textsf{FP}$ is not negligible, there does not exist an \textsf{RH} that is unforgeable for the distribution $\{ \mathcal{B} \getsdollar \mathbb{B}: \mathbf{b} \gets {\sf getEnroll}^{\mathcal{O}_\mathcal{B}}() \mid \mathbb{B} \}$.


Note that since $\textsf{esk}, \textsf{psk}$, and $ \textsf{csk}$ are all public in this instantiation, it is meaningless to discuss $\mathcal{O}_\textsf{Enroll}, \mathcal{O}_\textsf{Probe}$, or $\mathcal{O}_\textsf{log}$. In addition, for $\textsf{option}$ that includes $\mathcal{O}_\mathcal{B}$ or $\mathcal{O}_\textsf{Probe}^\prime$, as discussed in Section \ref{sec:uf_game}, we cannot achieve \textsf{option}-UF security since $\textsf{psk}$ is public in this instantiation.

For \textsf{option} that includes $\mathcal{O}_\textsf{Enroll}^\prime$, we notice that for the \textsf{RH} construction in \cite{cryptoeprint:2014/394}, there exists an invalid $\textsf{pk}^\prime$ such that $\textsf{RH.Hash}_1(\textsf{pk}^\prime, \mathbf{x})$ directly leaks $\mathbf{x}$. By returning $\textsf{RH.Hash}_2( \textsf{pk}, \mathbf{x} )$, one can break the $\textsf{UF}_{\textsf{option}}$ game with probability $1$.


%-------------------

\subsection{IND Security of $\Pi$}
\label{sec:security_analysis:rh:IND}

For the IND security, we have a negative result for $\Pi$.

\begin{theorem}

For any distribution family $\mathbb{B}$ that $\textsf{TP} - \textsf{FP} > \frac{1}{\poly}$, and for any relational hash scheme \textsf{RH}, $\Pi$ is not IND for any $t \geq 0$.

\end{theorem}

\begin{proof}

Consider the adversary $\mathcal{A}$ in Algorithm \ref{alg:rh:ind}. When the challenge bit $b = 0$, the probability that $\mathcal{A}$ wins is $\textsf{TP}$. When the challenge bit $b = 1$, the probability that $\mathcal{A}$ wins is $1 - \textsf{FP}$. Now, 

\[
	\Adv_{\Pi, \mathbb{B}, \mathcal{A}}^{\textsf{IND}} = \left| \Pr[ \textsf{IND}_\Pi(\mathcal{A}) \to 1 ] - \frac{1}{2} \right| = \left| \frac{1}{2} (\textsf{TP} + 1 - \textsf{FP}) - \frac{1}{2} \right| > \frac{1}{\poly}.
\]

\begin{figure}[h]
\centering

	\begin{minipage}[t]{0.65\textwidth}
	\begin{algorithm}[H]
	\caption{$\mathcal{A}^{\mathcal{O}_{\mathcal{B}^{(0)}}, \mathcal{O}_{\mathcal{B}^{(1)}}} ( \textsf{csk} = \textsf{pk}, \mathbf{c_x}, \{ \mathbf{c_y}^{(i)} \}_{i=1}^t )$}
	\label{alg:rh:ind}
	\begin{algorithmic}[1]

		\State $\mathbf{y}^{(0)} = \mathbf{b}^{(0)} \gets \textsf{getProbe}^{\mathcal{O}_{\mathcal{B}^{(0)}}}()$
		
		\State $\mathbf{h_y}^{(0)} \gets \textsf{RH.Hash}_2(\textsf{pk}, \mathbf{y}^{(0)})$
		
		\If{ $\textsf{RH.Verify}(\textsf{pk}, \mathbf{c_x}, \mathbf{h_y}^{(0)}) = 1$}
		
			\State \Return $0$
		
		\Else
			
			\State \Return $1$
		
		\EndIf
	\end{algorithmic}
	\end{algorithm}
	\end{minipage}

\end{figure}

\end{proof}

We note that this insecurity result holds whenever $\textsf{psk}$ is public. When $\textsf{esk}$ is public, one can also use $\mathbf{c_y}^{(i)}$ to verify from which distribution the chalenge ciphertexts are generated. We write this observation formally in the following theorem.

\begin{theorem}

Given any distribution family $\mathbb{B}$ that $\textsf{TP} - \textsf{FP} > \frac{1}{\poly}$. If $\textsf{psk}$ is public, $\Pi$ is not IND for any $t \geq 0$. If $\textsf{esk}$ is public, $\Pi$ is not IND for any $t \geq 1$.

\end{theorem}

\subsubsection{IND Security for a Particular Biometric Layer}

Recall that in Section \ref{sec:ind_game}, we introduce as an example a particular biometric layer:
\[
	\textsf{getEnroll}^{\mathcal{O}_{\mathcal{B}}}() \to \mathbf{b}^* + \mathcal{E}_{\textsf{Enroll}}  \quad \text{and} \quad \textsf{getProbe}^{\mathcal{O}_{\mathcal{B}}}() \to \mathbf{b}^* + \mathcal{E}_{\textsf{Probe}}
\]
where $\mathbf{b}^* \in \{0, 1\}^k$ is a fixed vector only dependent on $\mathcal{B}$, and $\mathcal{E}_{\textsf{Enroll}}, \mathcal{E}_{\textsf{Probe}} \subseteq \{0, 1\}^k$ are some \emph{error distributions} independent of $\mathcal{B}$.
With the same relational hash \textsf{RH} in Section \ref{sec:rh-instantiation}, we can instantiate another authentication scheme using \textsf{RH}.

\begin{itemize}

	\item $\textsf{Setup}(1^\lambda)$: It runs $\textsf{RH.KeyGen}(1^\lambda) \to \textsf{pk}$ and samples $\mathbf{r} \getsdollar \{0, 1\}^k$. Then it outputs $\textsf{esk} \gets (\textsf{pk}, \mathbf{r})$, $(\textsf{psk} \gets \textsf{pk}, \mathbf{r})$, and $\textsf{csk} \gets \textsf{pk}$.

	\item $\textsf{Enroll}(\textsf{esk}, \mathbf{b})$: Let $\mathbf{x} \gets \mathbf{b}$. It calls $\textsf{RH.Hash}_1(\textsf{pk}, \mathbf{x} + \mathbf{r}) \to \mathbf{h_x}$ and outputs $\mathbf{c_x} \gets \mathbf{h_x}$.

	\item $\textsf{Probe}(\textsf{psk}, \mathbf{b}^\prime)$: Let $\mathbf{y} \gets \mathbf{b}$. It calls $\textsf{RH.Hash}_2(\textsf{pk}, \mathbf{y} + \mathbf{r}) \to \mathbf{h_y}$ and outputs $\mathbf{c_y} \gets \mathbf{h_y}$.

	\item $\textsf{Compare}(\textsf{csk}, \mathbf{c_x}, \mathbf{c_y)}$: It calls $\textsf{RH.Verify}(\textsf{pk}, \mathbf{h_x}, \mathbf{h_y}) \to s$ and outputs the value $s$.

\end{itemize}
Correctness holds because
\[
	\textsf{Compare}(\textsf{csk}, \mathbf{c_x}, \mathbf{c_y)} = 1 \Leftrightarrow \textsf{HD}(\mathbf{x} + \mathbf{r}, \mathbf{y} + \mathbf{r}) \leq \tau \Leftrightarrow \textsf{HD}(\mathbf{x}, \mathbf{y}) \leq \tau = \textsf{BioCompare}(\mathbf{b}, \mathbf{b}^\prime).
\]

With the same argument in Theorem \ref{thm:rh:ind:particular-biometri-layer}, one can prove that this new scheme is now IND, albeit at the cost of requiring $\textsf{esk}$ and $\textsf{psk}$ to remain secret.
