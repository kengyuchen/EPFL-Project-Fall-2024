%%%%%%%%%%%%%%%%%%%%

% Project Name: Semester Project Fall 2024 for EPFL
% File: security_game.tex
% Author: Keng-Yu Chen

%%%%%%%%%%%%%%%%%%%%

In this section, we discuss two security notions of a biometric authentication scheme: \emph{unforgeability} and \emph{indistinguishability}.

\subsection{Unforgeability}
\label{sec:uf_game}

To describe the unforgeability of an authentication scheme, we model the ability of an adversary who tries to impersonate a user. The adversary $\mathcal{A}$ is given auxiliary information \textsf{option} that depends on our threat model and tries to find a valid probe message $\mathbf{\tilde{z}}$. The whole game $\textsf{UF}_{\Pi, \mathbb{B}, \textsf{option}}$ is defined in Algorithm \ref{alg:uf_game}.

\begin{figure}[h]
\centering
	\begin{minipage}[t]{0.6\linewidth}
	\centering
	\begin{algorithm}[H]
	\caption{$\textsf{UF}_{\Pi, \mathbb{B}, \textsf{option}}(\mathcal{A})$}
	\label{alg:uf_game}
	\begin{algorithmic}[1]
		\State $\mathcal{B} \getsdollar \mathbb{B}, \mathbb{B} \gets \mathbb{B} \setminus \mathcal{B}$

		\State $\textsf{esk}, \textsf{psk}, \textsf{csk} \gets \textsf{Setup}(1^\lambda)$
		
		\State $\mathbf{b} \gets \textsf{getEnroll}^{\mathcal{O}_{\mathcal{B}}}()$

		\State $\mathbf{c_x} \gets \textsf{Enroll}(\textsf{esk}, \mathbf{b})$
		
		\State ${\mathbf{\tilde{z}}} \gets \mathcal{A} ( \textsf{option} )$
 
		\If{$\mathbf{\tilde{z}}$ is equal to any output of $\mathcal{O}_{\textsf{Probe}}$ }
			
			\State \Return $0$
		
		\EndIf

		\State $s \gets \textsf{Compare}( \textsf{csk}, \mathbf{c_x}, \mathbf{\tilde{z}} )$

		\State \Return $\textsf{Verify}(s)$
	\end{algorithmic}
	\end{algorithm}
	\end{minipage}
	
\label{fig:uf_game}
\end{figure}

The auxiliary information \textsf{option} can be nothing or include $\mathbf{c_x}, \textsf{esk}, \textsf{psk}, \textsf{csk}$ or the following oracles:

\begin{itemize}

	\item $\mathcal{O}_{\mathcal{B}}$: It outputs a biometric sample $\mathbf{b} \getsdollar \mathcal{B}$. This oracle and $\textsf{psk}$ should not be given at the same time; otherwise, there exists a trivial attack with a winning rate $\textsf{TP}$ by returning $\textsf{Probe}(\textsf{psk}, \textsf{getProbe}^{ \mathcal{O}_\mathcal{B} }() )$.
	
	\item $\mathcal{O}_\textsf{Enroll}(\textsf{esk}, \cdot)$: On input $\mathbf{b}^\prime$, it outputs the enrollment message $\textsf{Enroll}(\textsf{esk}, \mathbf{b}^\prime)$.

	\item $\mathcal{O}_\textsf{Probe}(\textsf{psk}, \cdot)$: On input $\mathbf{b}^\prime$, it outputs the probe message $\textsf{Probe}(\textsf{psk}, \mathbf{b}^\prime)$. If this oracle is given, we require the adversary to return a $\mathbf{\tilde{z}}$ that is not equal to any previous answer of $\mathcal{O}_\textsf{Probe}$.
	
	\item $\mathcal{O}_\textsf{log}(\textsf{csk}, \mathbf{c_x}, \cdot)$: On input $\mathbf{b}^\prime$, it first computes $\mathbf{c_z} \gets \textsf{Probe}(\textsf{psk}, \mathbf{b}^\prime)$ and outputs $\textsf{Verify}(\textsf{Compare}(\textsf{csk}, \mathbf{c_x}, \mathbf{c_z} ) )$.
	
	\item $\mathcal{O}_\textsf{Enroll}^\prime (\cdot)$: On input $\textsf{esk}^\prime$, it first samples $\mathbf{b}^\prime \gets \textsf{getEnroll}^{\mathcal{O}_{\mathcal{B}}}()$ and outputs $\textsf{Enroll}(\textsf{esk}^\prime, \mathbf{b}^\prime)$. This oracle is only useful when $\textsf{option}$ does not include $\mathcal{O}_{\mathcal{B}}$.

	\item $\mathcal{O}_\textsf{Probe}^\prime (\cdot)$: On input $\textsf{psk}^\prime$, it first samples $\mathbf{b}^\prime \gets \textsf{getProbe}^{\mathcal{O}_{\mathcal{B}}}()$ and outputs $\textsf{Probe}(\textsf{psk}^\prime, \mathbf{b}^\prime)$. This oracle is only useful when $\textsf{option}$ does not include $\mathcal{O}_{\mathcal{B}}$, and this oracle and $\textsf{psk}$ should not be given at the same time; otherwise, there exists a trivial attack with a winning rate $\textsf{TP}$ by returning $\mathcal{O}_{\textsf{Probe}}^\prime (\textsf{psk})$.
	
\end{itemize}

The requirement that the adversary should return a $\mathbf{\tilde{z}}$ that is not equal to any previous answer of $\mathcal{O}_\textsf{Probe}$ is to prevent a trivial attack that leverages \textsf{TP} or \textsf{FP} when it is non-negligible. If \textsf{option} includes $\mathcal{O}_\mathcal{B}$ and either $\textsf{psk}$ or $\mathcal{O}_\textsf{Probe}$, the adversary can enjoy a winning rate \textsf{TP}. Therefore, we rule out the case that $\textsf{option}$ includes both $\textsf{psk}$ and $\mathcal{O}_\mathcal{B}$, and we forbid the adversary to return what $\mathcal{O}_\textsf{Probe}$ returns.
If \textsf{option} has only $\textsf{psk}$ or $\mathcal{O}_\textsf{Probe}$, the $\textsf{UF}$ adversary $\mathcal{A}$ in Algorithm \ref{alg:adv:FP} can still enjoy a winning rate $\textsf{FP}$, if we place no restriction on the adversary's answer. Therefore, we only consider $\textsf{psk}$ in \textsf{option} when \textsf{FP} is non-negligible, and we restrict the adversary's answer when $\mathcal{O}_\textsf{Probe}$ is given.

%If $\textsf{option}$ includes $\mathcal{O}_{\textsf{Enroll}}^\prime$ and either $\textsf{psk}$ or $\mathcal{O}_\textsf{Probe}$, and if we place no restriction on an \textsf{UF} game adversary's answer, the adversary in Algorithm \ref{alg:adv:FP2} can win with a probability
%\[
	%\Pr[ \textsf{Verify}( \textsf{BioCompare}(\mathbf{x}^{(0)}, \mathbf{y}) ) = 1 \mid \textsf{Verify}( \textsf{BioCompare}(\mathbf{x}^{(1)}, \mathbf{y}) ) = 1]
%\]
%where $\mathbf{x}^{(0)}, \mathbf{x}^{(1)}$ are generated from $\textsf{encodeEnroll}^{\mathcal{O}_\mathcal{B}}()$ and $\mathbf{y} \gets \textsf{encodeProbe}^{\mathcal{O}_{\mathcal{B}^\prime}}() $.
%This value is in general not negligible.
%The expected number of repetitions is $\mathbb{E}_{\mathcal{B} \getsdollar \mathbb{B}}\left[\frac{1}{\textsf{FP}(\mathcal{B})} \right]$. If $\textsf{FP}(\mathcal{B})$ is non-negligible, the adversary can return the answer in an expected polynomial time. A similar adversary also exists when $\textsf{option}$ includes $\mathcal{O}_{\textsf{Probe}}^\prime$ and $\mathcal{O}_\textsf{Probe}$.

 
%\begin{figure}[h]
%\centering
	%\begin{minipage}[t]{0.6\linewidth}
	%\centering
	%\begin{algorithm}[H]
	%\caption{$\mathcal{A}^{\mathcal{O}_\textsf{Enroll}^\prime}(\textsf{psk})$ (or  $\mathcal{A}^{\mathcal{O}_\textsf{Enroll}^\prime, \mathcal{O}_\textsf{Probe}}$ ) }
	%\label{alg:adv:FP2}
	%\begin{algorithmic}[1]
		%\State $\textsf{esk}^\prime, \textsf{psk}^\prime, \textsf{csk}^\prime \gets \textsf{Setup}(1^\lambda)$

		%\Repeat
		
			%\State $\mathcal{B}^\prime \getsdollar \mathbb{B}$
		
			%\State $\mathbf{y}^\prime \gets \textsf{encodeProbe}^{\mathcal{O}_{\mathcal{B}^\prime }}()$ 

			%\State $\mathbf{c_y}^\prime \gets \textsf{Probe}(\textsf{psk}^\prime, \mathbf{y}^\prime)$

			%\State $\mathbf{c_x}^\prime \gets \mathcal{O}_\textsf{Enroll}^\prime (\textsf{esk}^\prime)$

		%\Until{ $\textsf{Verify}(\textsf{Compare}(\textsf{csk}^\prime, \mathbf{c_x}^\prime, \mathbf{c_y}^\prime )) = 1$ }

		%\State $\mathbf{c_y} \gets \textsf{Probe}(\textsf{psk}, \mathbf{y}^\prime)$  \Comment{ or $\mathbf{c_y} \gets \mathcal{O}_\textsf{Probe}(\mathbf{y}^\prime)$ } 

		%\State \Return $\mathbf{c_y}$
	%\end{algorithmic}
	%\end{algorithm}
	%\end{minipage}
	
%\end{figure}

\begin{figure}[h]
\centering
	\begin{minipage}[t]{0.6\linewidth}
	\centering
	\begin{algorithm}[H]
	\caption{$\mathcal{A}(\textsf{psk})$ ( or $\mathcal{A}^{\mathcal{O}_\textsf{Probe}}$ ) }
	\label{alg:adv:FP}
	\begin{algorithmic}[1]
		\State $\mathcal{B}^\prime \getsdollar \mathbb{B}$
		
		\State $\mathbf{b}^\prime \gets \textsf{getProbe}^{\mathcal{O}_{\mathcal{B}^\prime }}()$

		\State $\mathbf{c_y} \gets \textsf{Probe}(\textsf{psk}, \mathbf{b}^\prime)$ \Comment{or $\mathbf{c_y} \gets \mathcal{O}_\textsf{Probe}(\mathbf{b}^\prime)$ }

		\State \Return $\mathbf{c_y}$
	\end{algorithmic}
	\end{algorithm}
	\end{minipage}
	
\end{figure}


We define the advantage of an adversary $\mathcal{A}$ in the $\textsf{UF}_{\Pi, \mathbb{B}, \textsf{option}}$ game of a scheme $\Pi$ associated with a family $\mathbb{B}$ of distributions as
\[
	\Adv^{\textsf{UF}}_{\Pi, \mathbb{B}, \mathcal{A}, \textsf{option}} := \Pr[\textsf{UF}_{\Pi, \mathbb{B}, \textsf{option}}(\mathcal{A}) \to 1]
\]

An authentication scheme $\Pi$ associated with a family $\mathbb{B}$ of distributions is called \emph{\textsf{option}-unforgeable} (\textsf{option}-UF) if for any PPT adversary $\mathcal{A}$,
\[
	\Adv^{\textsf{UF}}_{\Pi, \mathbb{B}, \mathcal{A}, \textsf{option}} = \negl.
\]

For the rest of this work, if the scheme $\Pi$, the family $\mathbb{B}$ of distributions, and the auxiliary information $\textsf{option}$ are clear from context, we omit the subscript and write the game as $\textsf{UF}(\mathcal{A})$. This abbreviation also holds for all other games.


\paragraph{UF Security with Digital Signature}

We note that we can achieve UF security by a similar approach in \cite{cryptoeprint:2023/481} with a digital signature scheme. Given any authentication scheme $\Pi$ and an sEUF-CMA digital signature scheme $\textsf{Sig} = (\textsf{Sig.KeyGen}, \textsf{Sig.Sign}, \textsf{Sig.Verify})$, consider the following scheme $\Pi^\prime$.

\begin{itemize}

	\item $\textsf{Setup}^\prime (1^\lambda)$: Run $(\textsf{esk}, \textsf{psk}, \textsf{csk}) \gets \textsf{Setup}(1^\lambda)$ and $(\textsf{sk}_{\textsf{Sig}}, \textsf{pk}_{\textsf{Sig}}) \gets \textsf{Sig.KeyGen}(1^\lambda)$. Output $\textsf{esk}^\prime \gets \textsf{esk}$, $\textsf{psk}^\prime \gets (\textsf{psk}, \textsf{sk}_{\textsf{Sig}})$, $\textsf{csk}^\prime \gets \textsf{csk}$.

	\item $\textsf{Probe}^\prime (\textsf{psk}^\prime, \mathbf{b}^\prime)$: Run $\mathbf{c_y} \gets \textsf{Probe}(\textsf{psk}, \mathbf{b}^\prime)$ and $\sigma \gets \textsf{Sig.Sign}(\textsf{sk}_{\textsf{Sig}}, \mathbf{c_y})$. Output $\mathbf{c_y}^\prime \gets (\mathbf{c_y}, \sigma)$.

	\item $\textsf{Compare}^\prime (\textsf{csk}, \mathbf{c_x}, \mathbf{c_y}^\prime)$: If $\textsf{Sig.Verify}(\textsf{pk}_{\textsf{Sig}}, \mathbf{c_y}, \sigma) = 1$, output $\textsf{Compare}(\textsf{csk}, \mathbf{c_x}, \mathbf{c_y})$; otherwise, output $\bot$.

\end{itemize}

An $\textsf{UF}_\textsf{option}$ adversary has to forge a signature $\sigma$ to win the game, so the scheme is $\textsf{option}$-UF for any $\textsf{option}$ that does not include $\textsf{psk}$. 

\begin{theorem}
\label{thm:sEUF-CMA-esk-csk}
	Let $\textsf{option} = \{ \mathbf{c_x}, \textsf{esk}, \textsf{csk}, \mathcal{O}_\mathcal{B}, \mathcal{O}_{\textsf{Probe}} \}$. For any authentication scheme $\Pi$, $\Pi^\prime$ is $\textsf{option}$-UF. 
\end{theorem}

%-------------------


\subsection{Indistinguishability}
\label{sec:ind_game}


In the game of indistinguishability, we model the ability of an authentication server who tries to identify the user, which describes the privacy leakage of the scheme. The adversary $\mathcal{A}$ is given oracles to two biometric distributions $\mathcal{B}^{(0)}$ and $ \mathcal{B}^{(1)}$, the comparison key $\textsf{csk}$, an enrollment message $\mathbf{c_x}$, and a list of $t$ probe messages $\{ \mathbf{c_y}^{(i)} \}_{i=1}^t$. It tries to guess from either $\mathcal{B}^{(0)}$ or $ \mathcal{B}^{(1)}$ these messages are generated. The whole game is defined in Algorithm \ref{alg:ind_game}.

\begin{figure}[h]
\centering

	\begin{minipage}[t]{0.55\textwidth}
	\begin{algorithm}[H]
	\caption{$\textsf{IND}_{\Pi, \mathbb{B}}(\mathcal{A})$}
	\label{alg:ind_game}
	\begin{algorithmic}[1]
		\State $b \getsdollar \{0, 1\}$

		\State $\mathcal{B}^{(0)} \getsdollar \mathbb{B}, \quad \mathbb{B} \gets \mathbb{B} \setminus \mathcal{B}^{(0)}$

		\State $\mathcal{B}^{(1)} \getsdollar \mathbb{B}, \quad \mathbb{B} \gets \mathbb{B} \setminus \mathcal{B}^{(1)}$

		\State $\textsf{esk}, \textsf{psk}, \textsf{csk} \gets \textsf{Setup}(1^\lambda)$

		\State $\mathbf{b} \gets \textsf{getEnroll}^{\mathcal{O}_{\mathcal{B}^{(b)}}}()$

		\State $\mathbf{c_x} \gets \textsf{Enroll}(\textsf{esk}, \mathbf{b})$

		\For{$i = 1$ to $t$}

			\State ${\mathbf{b}^\prime}^{(i)} \gets \textsf{getProbe}^{\mathcal{O}_{\mathcal{B}^{(b)}}}() $
		
			\State $\mathbf{c_y}^{(i)} \gets \textsf{Probe}( \textsf{psk}, {\mathbf{b}^\prime}^{(i)} )$

		\EndFor

		%\State In Device-of-User Model:
		
			%\State \hspace{\algorithmicindent} $\tilde{b} \gets \mathcal{A}^{\mathcal{O}_{\mathcal{B}^{(0)}}, \mathcal{O}_{\mathcal{B}^{(1)}} } ( \textsf{csk}, \mathbf{c_x}, \{ \mathbf{c_y}^{(i)} \}_{i=1}^t )$

		%\State In Device-of-Domain Model:
		
			\State $\tilde{b} \gets \mathcal{A}^{\mathcal{O}_{\mathcal{B}^{(0)}}, \mathcal{O}_{\mathcal{B}^{(1)}}} ( \textsf{csk}, \mathbf{c_x}, \{ \mathbf{c_y}^{(i)} \}_{i=1}^t )$

		\State \Return $1_{\tilde{b} = b}$
	\end{algorithmic}
	\end{algorithm}
	\end{minipage}

%\caption{The \textsf{IND} Game}
\label{fig:ind_game}
\end{figure}

The auxiliary information \textsf{option} can be nothing or include $\mathbf{c_x}, \textsf{csk}$ or the following oracles:

\begin{itemize}

	\item $\mathcal{O}_{\mathbf{x}}$: It first samples a biometric sample $\mathbf{b} \getsdollar \textsf{getEnroll}^{\mathcal{B}^{(b)} }()$ and outputs $\mathbf{c_x} \gets \textsf{Enroll}(\textsf{esk}, \mathbf{b})$. 
	
	\item $\mathcal{O}_{\mathbf{y}}$: It first samples a biometric sample $\mathbf{b}^{\prime} \getsdollar \textsf{getProbe}^{\mathcal{B}^{(b)} }()$ and outputs $\mathbf{c_y} \gets \textsf{Probe}(\textsf{psk}, \mathbf{b}^{\prime})$.

	\item $\mathcal{O}_\textsf{Enroll}(\textsf{esk}, \cdot)$: On input $\mathbf{b}^\prime$, it outputs the enrollment message $\textsf{Enroll}(\textsf{esk}, \mathbf{b}^\prime)$.

	\item $\mathcal{O}_\textsf{Probe}(\textsf{psk}, \cdot)$: On input $\mathbf{b}^\prime$, it outputs the probe message $\textsf{Probe}(\textsf{psk}, \mathbf{b}^\prime)$. If this oracle is given, we require the adversary to return a $\mathbf{\tilde{z}}$ that is not equal to any previous answer of $\mathcal{O}_\textsf{Probe}$.
	
\end{itemize}

We define the advantage of an adversary $\mathcal{A}$ in the \textsf{IND} game of a scheme $\Pi$ associated with a family of distributions $\mathbb{B}$ as
\[
	\Adv^{\textsf{IND}}_{\Pi, \mathbb{B}, \mathcal{A}} := \left |\Pr[\textsf{IND}_{\Pi}(\mathcal{A}) \to 1] - \frac{1}{2} \right|.
\]

An authentication scheme $\Pi$ associated with a family $\mathbb{B}$ of distributions is called \emph{indistinguishable (IND)} if for any PPT adversary $\mathcal{A}$,
\[
	\Adv^{\textsf{IND}}_{\Pi, \mathbb{B}, \mathcal{A}} = \negl.
\]

\paragraph{Necessity of IND Security}
Consider the following authentication scheme for any biometric layer. Let $\textsf{esk} = \textsf{psk} = \textsf{csk}$ be empty strings and
\begin{gather*}
	\textsf{Enroll}(\textsf{esk}, \mathbf{b}) \to \mathbf{b}, \quad \textsf{Probe}(\textsf{psk}, \mathbf{b}^\prime ) \to \mathbf{b}^\prime \\
	\textsf{Compare}(\textsf{csk}, \mathbf{b}, \mathbf{b}^\prime) = \textsf{BioCompare}(\mathbf{b}, \mathbf{b}^\prime ) 
\end{gather*}
By the transformation using an sEUF-CMA digital signature scheme we introduced in Section \ref{sec:uf_game}, we can obtain an authentication scheme $\Pi^\prime$ that is $\textsf{option}$-UF for any $\textsf{option}$ that does not include $\textsf{psk}$.
However, the enrollment and probe messages leak biometric vectors $\mathbf{b}$ and $\mathbf{b}^\prime$ and compromise privacy. Obviously, this scheme is not IND, and we use this example emphasize the necessity of the game of indistinguishability.


\paragraph{IND Security for a Particular Biometric Layer}

Let $\textsf{getEnroll}^{\mathcal{O}_{\mathcal{B}}}(), \textsf{getProbe}^{\mathcal{O}_{\mathcal{B}}}()$ be such that
\[
	\textsf{getEnroll}^{\mathcal{O}_{\mathcal{B}}}() \to \mathbf{b}^*_{\mathcal{B}} \oplus \mathcal{E}_{\textsf{Enroll}}  \quad \text{and} \quad \textsf{getProbe}^{\mathcal{O}_{\mathcal{B}}}() \to \mathbf{b}^*_{\mathcal{B}} \oplus \mathcal{E}_{\textsf{Probe}}
\]
where $\mathbf{b}^*_{\mathcal{B}} \in \{0, 1\}^k$ is a fixed vector only dependent on $\mathcal{B}$, and $\mathcal{E}_{\textsf{Enroll}}, \mathcal{E}_{\textsf{Probe}} \subseteq \{0, 1\}^k$ are some \emph{error distributions} independent of $\mathcal{B}$. Let $\textsf{BioCompare}(\mathbf{b}, \mathbf{b}^\prime) \to 1_{\textsf{HD}(\mathbf{b}, \mathbf{b}^\prime) \leq \tau}$. Then
\[
	\textsf{TP} = \Pr[ \textsf{HW}(\mathbf{b}^*_{\mathcal{B}} \oplus \mathcal{E}_{\textsf{Enroll}} \oplus \mathbf{b}^*_{\mathcal{B}} \oplus \mathcal{E}_{\textsf{Probe}}) \leq \tau ] = \Pr[ \textsf{HW}(\mathcal{E}_{\textsf{Enroll}} \oplus \mathcal{E}_{\textsf{Probe}}) \leq \tau ]
\]

\noindent We note that previous works such as \cite{10.1145/1030083.1030096,cryptoeprint:2014/394} model biometric template vectors in a similar way.

Now, for this biometric layer, we can construct a simple but IND secure authentication scheme $\Pi$ with the following cryptographic layer.

\begin{itemize}

	\item $\textsf{Setup} (1^\lambda)$: Sample $\mathbf{r} \getsdollar \{0, 1\}^k$. Output $\textsf{esk} = \textsf{psk} \gets \mathbf{r}$, $\textsf{csk} \gets \epsilon$.

	\item $\textsf{Enroll}(\textsf{esk}, \mathbf{b})$: Output $\mathbf{b} \oplus \mathbf{r}$.
	
	\item $\textsf{Probe}(\textsf{psk}, \mathbf{b}^\prime)$: Output $\mathbf{b}^\prime \oplus \mathbf{r}$.

	\item $\textsf{Compare} (\textsf{csk}, \mathbf{c_x}, \mathbf{c_y})$: If $\textsf{HD}(\mathbf{c_x}, \mathbf{c_y}) \leq \tau$, return $1$; otherwise, return $0$. 

\end{itemize}
The correctness of $\Pi$ holds by
\[
	\textsf{HD}(\mathbf{c_x}, \mathbf{c_y}) = \textsf{HW}(\mathbf{b} \oplus \mathbf{r} \oplus \mathbf{b}^\prime \oplus \mathbf{r}) = \textsf{HW}(\mathbf{b} \oplus \mathbf{b}^\prime) = \textsf{BioCompare}(\mathbf{b}, \mathbf{b}^\prime).
\]

\begin{theorem}
\label{thm:rh:ind:particular-biometri-layer}
The authentication scheme $\Pi$ is IND.

\end{theorem}

\begin{proof}

Let $\mathbf{b}^*_0$ and $\mathbf{b}^*_1$ be the fixed vectors of $\mathcal{B}^{(0)}$ and $\mathcal{B}^{(1)}$ in the \textsf{IND} game, respectively. For any $\mathbf{v}, \mathbf{v}^{(1)}, \cdots, \mathbf{v}^{(t)}$,
\begin{align*}
	& \Pr[ \mathbf{c_x} = \mathbf{v},\; \mathbf{c_y}^{(1)} = \mathbf{v}^{(1)},\; \cdots,\; \mathbf{c_y}^{(t)} = \mathbf{v}^{(t)} \mid b = 0,\; \mathbf{b}^*_0,\; \mathbf{b}^*_1 ] \\
	&= \Pr[ \mathbf{b}^*_0 \oplus \mathcal{E}_{\textsf{Enroll}} \oplus \mathbf{r} = \mathbf{v},\; \mathbf{b}^*_0 \oplus \mathcal{E}_{\textsf{Probe}} \oplus \mathbf{r} =\mathbf{v}^{(1)},\; \cdots,\; \mathbf{b}^*_0 \oplus \mathcal{E}_{\textsf{Probe}} \oplus \mathbf{r} =\mathbf{v}^{(t)} \mid \mathbf{b}^*_0,\; \mathbf{b}^*_1] \\ 
	&= \Pr[ \mathbf{r} = \mathbf{v} \oplus \mathbf{b}^*_0 \oplus \mathcal{E}_{\textsf{Enroll}} =\mathbf{v}^{(1)} \oplus \mathbf{b}^*_0 \oplus \mathcal{E}_{\textsf{Probe}} = \cdots = \mathbf{v}^{(t)} \oplus \mathbf{b}^*_0 \oplus \mathcal{E}_{\textsf{Probe}} \mid \mathbf{b}^*_0,\; \mathbf{b}^*_1] \\ 
	&= \Pr[ \mathbf{r} = \mathbf{v} \oplus \mathbf{b}^*_1 \oplus \mathcal{E}_{\textsf{Enroll}} = \mathbf{v}^{(1)} \oplus \mathbf{b}^*_1 \oplus \mathcal{E}_{\textsf{Probe}} = \cdots = \mathbf{v}^{(t)} \oplus \mathbf{b}^*_1 \oplus \mathcal{E}_{\textsf{Probe}} \mid \mathbf{b}^*_0,\; \mathbf{b}^*_1] \\ 
	&= \Pr[ \mathbf{b}^*_1 \oplus \mathcal{E}_{\textsf{Enroll}} \oplus \mathbf{r} = \mathbf{v},\; \mathbf{b}^*_1 \oplus \mathcal{E}_{\textsf{Probe}} \oplus \mathbf{r} =\mathbf{v}^{(1)},\; \cdots,\; \mathbf{b}^*_1 \oplus \mathcal{E}_{\textsf{Probe}} \oplus \mathbf{r} =\mathbf{v}^{(t)} \mid \mathbf{b}^*_0,\; \mathbf{b}^*_1] \\ 
	&= \Pr[ \mathbf{c_x} = \mathbf{v},\; \mathbf{c_y}^{(1)} = \mathbf{v}^{(1)},\; \cdots,\; \mathbf{c_y}^{(t)} = \mathbf{v}^{(t)} \mid b = 1,\; \mathbf{b}^*_0,\; \mathbf{b}^*_1  ]
\end{align*}
Hence, the adversary cannot distinguish between $\mathbf{c_x}, \mathbf{c_y}^{(1)}, \cdots, \mathbf{c_y}^{(t)}$ generated from $\mathcal{B}^{(0)}$ and $\mathcal{B}^{(1)}$.

\end{proof}
