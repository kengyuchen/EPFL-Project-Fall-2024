%%%%%%%%%%%%%%%%%%%%

% Semester Project for EPFL
% Author: Keng-Yu Chen

%%%%%%%%%%%%%%%%%%%%

%%%%%%%%%%%%%%%%%%%%

% Semester Project for EPFL
% Author: Keng-Yu Chen

%%%%%%%%%%%%%%%%%%%%

\documentclass[a4paper, 12pt]{article}
\usepackage[utf8]{inputenc}
\usepackage[margin=3cm]{geometry}
\setlength{\headheight}{14.49998pt}

% Include Packages
\usepackage{mathrsfs}
\usepackage{color}
\usepackage{amsthm,amsmath,amssymb}
\usepackage{bm}
\usepackage{stmaryrd}
\usepackage{hyperref}
\usepackage{graphicx}
\usepackage{multicol}
\usepackage{wrapfig}
\usepackage{subcaption}
\hypersetup{colorlinks=true,linkcolor=blue,citecolor=blue,filecolor=magenta,urlcolor=blue}

\usepackage{fancyhdr}
\usepackage{thmtools}

\usepackage{algorithm}
\usepackage{algpseudocode}
\renewcommand{\algorithmicrequire}{\textbf{Input:}}
\renewcommand{\algorithmicensure}{\textbf{Output:}}



% Set Self-defined Words
\DeclareMathOperator*{\argmax}{arg\!\max}
\DeclareMathOperator*{\argmin}{arg\!\min}

\newcommand{\negl}{\mathsf{negl}}
\newcommand{\poly}{\mathsf{poly}}
\newcommand{\Adv}{\mathbf{Adv}}

\newcommand{\R}{\mathbb{R}}
\newcommand{\N}{\mathbb{N}}
\newcommand{\Z}{\mathbb{Z}}
\newcommand{\Q}{\mathbb{Q}}

\newcommand\getsdollar{\mathrel{{\leftarrow}\vcenter{\hbox{\tiny\rmfamily\upshape\$}}}}


% Convert block[name] into a theorem-like format
\newtheorem{theorem}{Theorem}

\theoremstyle{definition}
\newtheorem{definition}{Definition}

\theoremstyle{definition}
\newtheorem{assumption}{Assumption}

\theoremstyle{plain}
\newtheorem{corollary}{Corollary}



% TikZ Setting
\usepackage{tikz}
\usetikzlibrary{shapes.geometric, arrows}
\usetikzlibrary{positioning}
\usetikzlibrary{calc}
\usetikzlibrary{backgrounds}

\tikzstyle{st-arrow} = [thick,->,>=stealth]
\tikzstyle{arrow} = [thick,-latex']
\tikzstyle{darrow} = [thick,latex'-latex']

\tikzstyle{gadget} = [rectangle, rounded corners,
	minimum width=2cm,
	minimum height=1cm,
	text centered,
	draw=black
]

\tikzstyle{element} = [rectangle, rounded corners,
	minimum width=2cm,
	minimum height=1.2cm,
	text centered,
	fill=white!70!black,
	draw=black
]
\pgfdeclarelayer{background layer}
\pgfdeclarelayer{foreground layer}
\pgfsetlayers{background layer,main,foreground layer}




% Add Referemces
\usepackage[
	backend=biber,
	style=alphabetic,
	sorting=ynt
]{biblatex}
\addbibresource{reference.bib}


\title{\textbf{Biometrics Authentication: Formalization and Instantiation}}
\author{Keng-Yu Chen}

\date{\today}



\begin{document}

%% Title
\maketitle

%-------------------

%% Header and Foot
\pagestyle{fancy}
\fancyhf{}
\fancyhead[L]{Semester Project}
\fancyhead[C]{Biometric Authentication}
\fancyhead[R]{Keng-Yu Chen}
\fancyfoot[C]{\thepage}

%-------------------

This report formalizes the biometric authentication scheme, including its structure, usage, and security analysis with a security game model.

%-------------------

\section*{Preliminaries}
\label{sec:preliminaries}

In this report, we assume

\begin{itemize}
	
	\item $\lambda$ is the security parameter.

	\item $[m]$ denotes the set of integers $\{1, 2, \cdots, m\}$.

	\item $\Z_q$ is the finite field modulo a prime number $q$.

	\item A function $f(n)$ is called \emph{negligible} iff for any integer $c$, $f(n) < \frac{1}{n^c}$ for all sufficiently large $n$. We write it as $f(n) = \negl$, and we may also use $\negl$ to represent an arbitrary negligible function.
	
	\item $\poly$ is the class of polynomial funcions. We may also use $\poly$ to represent an arbitrary polynomial function.
	
	\item We write sampling a value $r$ from a distribution $\mathcal{D}$ as $r \getsdollar \mathcal{D}$. If $S$ is a finite set, then $r \getsdollar S$ means sampling $r$ uniformly from $S$.

	\item The distribution $\mathcal{D}^t$ denotes $t$ identical and independent distributions of $\mathcal{D}$.

	\item A PPT algorithm denotes a probabilistic polynomial time algorithm. Unless otherwise specified, all algorithms run in PPT.

\end{itemize}


\begin{definition}[Functional Hiding Inner Product Functional Encryption]
\label{def:fh-IPFE}
	A \emph{functional hiding inner product functional encryption} (fh-IPFE) scheme $\sf FE$ for a field $\mathbb{F}$ and input length $k$ is composed of PPT algorithms ${\sf FE.Setup}$, ${\sf FE.KeyGen}$, ${\sf FE.Enc}$, and ${\sf FE.Dec}$:

\begin{itemize}

	\item ${\sf FE.Setup}(1^\lambda) \to {\sf msk}, {\sf pp}$: It outputs the public parameter ${\sf pp}$ and the master secret key ${\sf msk}$.

	\item ${\sf FE.KeyGen}({\sf msk}, {\sf pp}, \mathbf{x}) \to f_\mathbf{x}$: It generates the functional decryption key $f_\mathbf{x}$ for an input vector $\mathbf{x} \in \mathbb{F}^k$. 

	\item ${\sf FE.Enc}({\sf msk}, {\sf pp}, \mathbf{y}) \to \mathbf{c_y}$: It encrypts the input vector $\mathbf{y} \in \mathbb{F}^k$ to the ciphertext $\mathbf{c_y}$. 

	\item ${\sf FE.Dec}({\sf pp}, f_\mathbf{x}, \mathbf{c_y}) \to z \in \mathbb{F}$: It outputs a value $z$.

\end{itemize}

\noindent Correctness: The fh-IPFE scheme {\sf FE} is \emph{correct} if $\forall {\sf msk}, {\sf pp} \gets {\sf FE.Setup}$, $\forall \mathbf{x}, \mathbf{y} \in \mathbb{F}^k$, we have
	\[
		{\sf FE.Dec}( {\sf pp}, {\sf FE.KeyGen}({\sf msk}, {\sf pp}, \mathbf{x}), {\sf FE.Enc}({\sf msk}, {\sf pp}, \mathbf{y}) ) = \langle \mathbf{x} , \mathbf{y} \rangle \in \mathbb{F}.
	\]

\end{definition}

%-------------------

\section*{Formalization}
\label{sec:formalization}


In general, an authentication shceme $\Pi$ associated with a family of biometric distributions $\mathbb{B}$ is composed of the following algorithms.

\begin{itemize}

	\item ${\sf USetup}(1^\lambda) \to {\sf esk}, {\sf psk}, {\sf csk}, {\sf pp}$: It outputs the enrollment secret key ${\sf esk}$, probe secret key ${\sf psk}$, compare secret key $\sf csk$, and public parameter ${\sf pp}$.

	\item ${\sf encodeEnroll}^{\mathcal{O}_{\sf aux}}({\mathbf{b}}) \to \mathbf{x}$: Given a biometric template $\mathbf{b}$ sampled from some distribution $\mathcal{B} \in \mathbb{B}$, it encodes $\mathbf{b}$ as $\mathbf{x}$, the input format for enrollment. It can query an auxiliary oracle $\mathcal{O}_{\sf aux}$ to ask for additional biometric information of $\mathcal{B}$. 

	\item ${\sf Enroll}({\sf esk}, {\sf pp}, \mathbf{x}) \to \mathbf{c_x}$: It outputs the enrollment message $\mathbf{c_x}$ from $\mathbf{x}$.

	\item ${\sf encodeProbe}^{\mathcal{O}_{\sf aux}}({\mathbf{b'}}) \to \mathbf{y}$: Given a biometric template $\mathbf{b}'$ sampled from some distribution $\mathcal{B} \in \mathbb{B}$, it encodes $\mathbf{b}'$ as $\mathbf{y}$, the input format for probe. It can query an auxiliary oracle $\mathcal{O}_{\sf aux}$ to ask for additional biometric information of $\mathcal{B}$. 

	\item ${\sf Probe}({\sf psk}, {\sf pp}, \mathbf{y}) \to \mathbf{c_y}$: It outputs the probe message $\mathbf{c_y}$ from $\mathbf{y}$.

	\item ${\sf Compare}({\sf csk}, {\sf pp}, \mathbf{c_x}, \mathbf{c_y)} \to s$: It compares the enrollment message $\mathbf{c_x}$ and probe message $\mathbf{c_y}$ and outputs a score $s$.

	\item ${\sf Verify}(s) \to r \in \{0,1\}$: It is a deterministic algorithm that reads the comparison score $s$ and determines whether this is a successful authentication ($r = 1$) or not ($r = 0$).

\end{itemize}


The usage model we consider is described in Figure \ref{table:model}. We assume the user's biometric distribution is $\mathcal{B} \in \mathbb{B}$.

\begin{figure}[]
	\begin{center}
		\begin{tabular}{l c r c}
			{\bf User} &	& {\bf Server} \\

			\hline

			\\

			${\sf esk}, {\sf psk}, {\sf csk}, {\sf pp} \gets {\sf USetup}(1^\lambda)$ \\

			$ \mathbf{b} \getsdollar \mathcal{B}, \mathbf{x} \gets {\sf encodeEnroll}^{\mathcal{O}_{\sf aux}}(\mathbf{b})$ \\

			$\mathbf{c_x} \gets {\sf Enroll}({\sf esk}, {\sf pp}, \mathbf{x})$ \\

			& $\xrightarrow{\quad {\sf csk}, {\sf pp}, \mathbf{c_x} \quad}$ \\

			$ \mathbf{b'} \getsdollar \mathcal{B}, \mathbf{y} \gets {\sf encodeProbe}^{\mathcal{O}_{\sf aux}}(\mathbf{b'})$ \\

			$\mathbf{c_y} \gets {\sf Probe}({\sf psk}, {\sf pp}, \mathbf{y})$ \\

			& $\xrightarrow{\quad \mathbf{c_y} \quad}$ \\
			
			&  & $s \gets {\sf Compare} ({\sf csk}, {\sf pp}, \mathbf{c_x}, \mathbf{c_y})$ \\

			& & $r \gets {\sf Verify}(s) $ \\

			& $\xleftarrow{\quad r \quad}$ & 
		\end{tabular}
	\end{center}
	\caption{Authentication Model with User and Server}
	\label{table:model}
\end{figure}


For example, let ${\sf FE} = ({\sf FE.Setup}, {\sf FE.KeyGen}, {\sf FE.Enc}, {\sf FE.Dec})$ be an fh-IPFE scheme we defined in Definition \ref{def:fh-IPFE}. Following \cite{cryptoeprint:2023/481}, we can instantiate a biometric authentication scheme using ${\sf FE}$ with the distance metric the Euclidean distance. Let the biometric templates $\mathbf{b}$ and $\mathbf{b'}$ be sampled from some distribution $\mathcal{B} \subseteq [m]^k$, and let the associated field of ${\sf FE}$ be $\mathbb{Z}_q$ where $q$ is a prime number larger than the maximum possible Euclidean distance $m^2 \cdot k$. The scheme is instantiated as follows.

\begin{itemize}

	\item ${\sf USetup}(1^\lambda)$: It calls ${\sf FE.Setup}(1^\lambda) \to {\sf msk}, {\sf pp}$ and outputs $({\sf esk}, {\sf psk}, {\sf pp}) \gets ({\sf msk}, {\sf msk}, {\sf pp})$ and ${\sf csk}$ an empty string.

	\item ${\sf encodeEnroll}^{\mathcal{O}_{\sf aux}}({\mathbf{b}}) \to \mathbf{x}$: For a template vector $\mathbf{b} = (b_1, b_2, \cdots, b_k)$, the function encodes it as $\mathbf{x} = (x_1, x_2, \cdots, x_{k+2}) = (b_1, b_2, \cdots, b_k, 1, \|\mathbf{b}\|^2)$. The auxiliary oracle is empty.

	\item ${\sf Enroll}({\sf esk}, {\sf pp}, \mathbf{x})$: It calls ${\sf FE.KeyGen}({\sf esk}, {\sf pp}, \mathbf{x}) \to f_\mathbf{x}$ and outputs $\mathbf{c_x} \gets f_\mathbf{x}$.

	\item ${\sf encodeProbe}^{\mathcal{O}_{\sf aux}}({\mathbf{b'}})$: For a template vector $\mathbf{b'} = (b_1', b_2', \cdots, b_k')$, the function encodes it as $\mathbf{y} = (y_1, y_2, \cdots, y_{k+2}) = (-2b_1', -2b_2', \cdots, -2b_k', \|\mathbf{b'}\|^2, 1)$. The auxiliary oracle is empty.

	\item ${\sf Probe}({\sf psk}, {\sf pp}, \mathbf{y})$: It calls ${\sf FE.Enc}({\sf psk}, {\sf pp}, \mathbf{y}) \to \mathbf{c_y}$ and outputs $\mathbf{c_y}$.

	\item ${\sf Compare}({\sf csk}, {\sf pp}, \mathbf{c_x}, \mathbf{c_y)}$: It calls ${\sf FE.Dec}({\sf pp}, \mathbf{c_x}, \mathbf{c_y}) \to s$ and outputs the value $s$.

	\item ${\sf Verify}(s)$: If $\sqrt{s} < \tau$, a pre-defined threshold for comparing the closeness of two templates, then it outputs $r = 1$; otherwise, it outputs $r = 0$.

\end{itemize}

By the correctness of the functional encryption scheme $\sf FE$, we have
\[
	s = {\sf FE.Dec}({\sf pp}, \mathbf{c_x}, \mathbf{c_y}) = \langle \mathbf{x}, \mathbf{y} \rangle = \sum_{i=1}^k -2b_ib_i' + \|\mathbf{b}\|^2 + \|\mathbf{b'}\|^2 = \| \mathbf{b} - \mathbf{b'} \|^2.
\]

which is the square of the Euclidean distance between two templates $\mathbf{b}$ and $\mathbf{b}'$. Therefore, if two templates $\mathbf{b}$ and $\mathbf{b}'$ are close enough such that $\|\mathbf{b} - \mathbf{b'}\| < \tau$, the scheme results in $r = 1$, a successful authentication.


%-------------------

\section*{Security Games}


\subsection*{Forgery Game}
\label{sec:forgery_game}

In the forgery game, we model the ability of an adversary who has access to the server's database of registered enrollments and tries to forge the user. The adversary $\mathcal{A}$ is given the enrollment message $\mathbf{c_x}$ and oracle $\mathcal{O}$ and tries to find a valid probe message $\mathbf{\tilde{z}}$. The whole game is defined in Figure \ref{fig:forgery_game}.

\begin{figure}[h]
\centerline{

\minipage{0.45\textwidth}
	\begin{tabular}{l c}
		${\sf Forg}_{\Pi, \mathbb{B}}(\mathcal{A}^\mathcal{O})$\\

			\hline

			$\mathcal{B} \getsdollar \mathbb{B}$ \\

			${\sf esk}, {\sf psk}, {\sf csk}, {\sf pp} \gets {\sf USetup}(1^\lambda)$ \\

			$ \mathbf{b} \getsdollar \mathcal{B}, \mathbf{x} \gets {\sf encodeEnroll}^{\mathcal{O}_{\sf aux}}(\mathbf{b})$ \\

			$\mathbf{c_x} \gets {\sf Enroll}({\sf esk}, {\sf pp}, \mathbf{x})$ \\

			${\mathbf{\tilde{z}}} \gets \mathcal{A}^{\mathcal{O}} ( {\sf pp}, \mathbf{c_x} )$ \\

			$s \gets {\sf Compare}( {\sf csk}, {\sf pp}, \mathbf{c_x}, \mathbf{\tilde{z}} )$ \\

			\textbf{return} ${\sf Verify}(s)$
			
	\end{tabular}
	\caption{The Forgery Game}
	\label{fig:forgery_game}
\endminipage
\minipage{0.45\textwidth}
	\begin{tabular}{l c}
		${\sf Forg'}_{\Pi, \mathbb{B}}(\mathcal{A}')$\\

			\hline

			$\mathcal{B} \getsdollar \mathbb{B}$ \\

			${\sf esk}, {\sf psk}, {\sf csk}, {\sf pp} \gets {\sf USetup}(1^\lambda)$ \\

			$ \mathbf{b} \getsdollar \mathcal{B}, \mathbf{x} \gets {\sf encodeEnroll}^{\mathcal{O}_{\sf aux}}(\mathbf{b})$ \\

			$\mathbf{c_x} \gets {\sf Enroll}({\sf esk}, {\sf pp}, \mathbf{x})$ \\

			${\mathbf{\tilde{z}}} \gets \mathcal{A}'({\sf pp})$ \\

			$s \gets {\sf Compare}( {\sf csk}, {\sf pp}, \mathbf{c_x}, \mathbf{\tilde{z}} )$ \\

			\textbf{return} ${\sf Verify}(s)$
			
		\end{tabular}
	\caption{The Plain Forgery Game}
	\label{fig:plain_forgery_game}
\endminipage
}
\end{figure}


The given oracle $\mathcal{O}$ can be any or more of the following three oracles:

\begin{itemize}

	\item $\mathcal{O}_{\sf Enroll}({\sf esk}, {\sf pp}, \cdot)$: On input $\mathbf{x}$, it outputs the enrollment message ${\sf Enroll}({\sf esk}, {\sf pp}, \mathbf{x})$.

	\item $\mathcal{O}_{\sf Probe}({\sf psk}, {\sf pp}, \cdot)$: On input $\mathbf{y}$, it outputs the probe message ${\sf Probe}({\sf psk}, {\sf pp}, \mathbf{y})$.

	\item $\mathcal{O}_{\sf Compare}({\sf csk}, {\sf pp}, \cdot, \cdot)$: On input $\mathbf{c_x}$ and $\mathbf{c_y}$, it outputs the comparison result ${\sf Compare}( {\sf csk}, {\sf pp}, \mathbf{c_x}, \mathbf{c_y} )$.

\end{itemize}

Note that if the enrollment secret key ${\sf esk}$, probe secret key ${\sf psk}$, or the comparison secret key ${\sf csk}$ is an empty string in the scheme, then the corresponding oracles are naturally and implicitly given since the adversary can compute them herself. 

To consider potential false positives of biometrics match, we consider the plain forgery game in Figure \ref{fig:plain_forgery_game}, in which the adversary does not have any knowledge about the template. 

We define the advantage of an adversary $\mathcal{A}$ in the forgery game of a scheme $\Pi$ associated with a family of distributions $\mathbb{B}$ as
\[
	\Adv^{\sf Forg}_{\Pi, \mathbb{B}, \mathcal{A}^\mathcal{O}} := \Pr[{\sf Forg}_{\Pi, \mathbb{B}}(\mathcal{A}^\mathcal{O}) \to 1] -
	\max_{\text{PPT } \mathcal{A}'} \Pr[{\sf Forg'}_{\Pi, \mathbb{B}}(\mathcal{A}') \to 1].
\]

An authentication scheme $\Pi$ associated with a family of distributions $\mathbb{B}$ is called \emph{forgery secure} if for any PPT adversary $\mathcal{A}$,
\[
	\Adv^{\sf Forg}_{\Pi, \mathbb{B}, \mathcal{A^\mathcal{O}}} = \negl.
\]

%-------------------

\subsection*{Simulation Game}
\label{sec:simulation_game}

In the simulation game, we model the ability of the server who tries to learn something more than the comparison result of the enrollment and probe messages. The adversary $\mathcal{A}$ is given an enrollment message and a list of $t$ probe messages, and she needs to guess whether these are real messages or simulation results of a simulator $\mathcal{S} = (\mathcal{S}_0, \mathcal{S}_{\sf Enroll}, \mathcal{S}_{\sf Probe})$ based on the $\sf Compare$ function. Intuitively, the simulator receives a list of $\sf Compare$ results of real enrollment and probe messages, and it returns some manual enrollment or probe messages that look similar to real ones. The whole game is defined in Figure \ref{fig:sim_game}.

\begin{figure}[h]
	\begin{center}

		\begin{subfigure}[t]{0.49\textwidth}
		\begin{tabular}{l c}
			${\sf SIM-Real}_{\Pi, \mathbb{B}}(\mathcal{A}^\mathcal{O})$\\

			\hline

			$\mathcal{B} \getsdollar \mathbb{B}$ \\

			${\sf esk}, {\sf psk}, {\sf csk}, {\sf pp} \gets {\sf USetup}(1^\lambda)$ \\

			$ \mathbf{b} \getsdollar \mathcal{B}, \mathbf{x} \gets {\sf encodeEnroll}^{\mathcal{O}_{\sf aux}}(\mathbf{b})$ \\

			$\mathbf{c_x} \gets {\sf Enroll}({\sf esk}, {\sf pp}, \mathbf{x})$ \\

			$\{ \mathbf{b}^{(i)} \}_{i=1}^t \getsdollar \mathcal{B}^t$ \\ 

			$\{\mathbf{y}^{(i)}\}_{i=1}^t \gets \{{\sf encodeProbe}^{\mathcal{O}_{\sf aux}}(\mathbf{b}^{(i)})\}_{i=1}^t $ \\
			
			$\{ \mathbf{c_y}^{(i)} \}_{i=1}^t \gets \left\{ {\sf Probe}( {\sf psk}, {\sf pp}, \mathbf{y}^{(i)} ) \right \}_{i=1}^t$ \\

			$b \gets \mathcal{A}^{\mathcal{O}} ({\sf csk}, {\sf pp}, \mathbf{c_x}, \{\mathbf{c_y}^{(i)}\}_{i=1}^t )$ \\

			\textbf{return} $b$
			
		\end{tabular}
		\end{subfigure}
		\begin{subfigure}[t]{0.49\textwidth}
		\begin{tabular}{l c}
			${\sf SIM-Ideal}_{\Pi, \mathbb{B}}(\mathcal{A}^{\tilde{\mathcal{O}}}, \mathcal{S} = (\mathcal{S}_0, \mathcal{S}_{\sf Enroll}, \mathcal{S}_{\sf Probe}))$\\

			\hline

			$\mathcal{B} \getsdollar \mathbb{B}$ \\

			${\sf esk}, {\sf psk}, {\sf csk}, {\sf pp} \gets {\sf USetup}(1^\lambda)$ \\

			$ \mathbf{b} \getsdollar \mathcal{B}, \mathbf{x} \gets {\sf encodeEnroll}^{\mathcal{O}_{\sf aux}}(\mathbf{b})$ \\

			$\mathbf{c_x} \gets {\sf Enroll}({\sf esk}, {\sf pp}, \mathbf{x})$ \\

			$\{ \mathbf{b}^{(i)} \}_{i=1}^t \getsdollar \mathcal{B}^t$ \\ 

			$\{\mathbf{y}^{(i)}\}_{i=1}^t \gets \{{\sf encodeProbe}^{\mathcal{O}_{\sf aux}}(\mathbf{b}^{(i)})\}_{i=1}^t $ \\
			
			$\{ \mathbf{c_y}^{(i)} \}_{i=1}^t \gets \left\{ {\sf Probe}( {\sf psk}, {\sf pp}, \mathbf{y}^{(i)} ) \right \}_{i=1}^t$ \\

			${\sf C} \gets \{ {\sf Compare}({\sf csk}, {\sf pp}, \mathbf{c_x}, \mathbf{c_y}^{(i)}) \}_{i=1}^t$ \\

			$\mathbf{c_x}', \{\mathbf{c_y}^{(i)'}\}_{i=1}^t \gets \mathcal{S}_0( {\sf csk}, {\sf C} )$ \\

			$b \gets \mathcal{A}^{\tilde{\mathcal{O}}} ({\sf csk}, {\sf pp}, \mathbf{c_x}', \{ \mathbf{c_y}^{(i)'} \}_{i=1}^t )$ \\

			\textbf{return} $b$
			
		\end{tabular}
		\end{subfigure}
	\end{center}
	\caption{The Simulation Game}
	\label{fig:sim_game}
\end{figure}

The oracle $\mathcal{O}$ can be any or more of the following oracles:

\begin{itemize}

	\item $\mathcal{O}_{\sf Enroll}({\sf esk}, {\sf pp}, \cdot)$: On input $\mathbf{x}$, it outputs the enrollment message ${\sf Enroll}({\sf esk}, {\sf pp}, \mathbf{x})$.

	\item $\mathcal{O}_{\sf Probe}({\sf psk}, {\sf pp}, \cdot)$: On input $\mathbf{y}$, it outputs the probe message ${\sf Probe}({\sf psk}, {\sf pp}, \mathbf{y})$.

\end{itemize}

The oracle $\tilde{\mathcal{O}}$ is stateful and memorizes all the previous queries. It includes two simulators $\mathcal{S}_{\sf Enroll}, \mathcal{S}_{\sf Probe}$ and is given by any or more of the following oracles:

\begin{itemize}

	\item $\tilde{\mathcal{O}}_{\sf Enroll} ({\sf csk} ,{\sf esk}, {\sf pp}, \cdot)$: On input $\mathbf{x}^{(j)}$, it updates the collection of comparison results ${\sf C}$ by adding the results with all previous probe messages $\mathbf{c_y}^{(i)}$. Then it calls the simulator $\mathcal{S}_{\sf Enroll}({\sf csk}, {\sf C})$ and returns whatever the simulator returns.

	\item $\tilde{\mathcal{O}}_{\sf Probe} ({\sf csk}, {\sf psk}, {\sf pp}, \cdot)$: On input $\mathbf{y}^{(j)}$, it updates the collection of comparison results ${\sf C}$ by adding the results with all previous enrollment messages $\mathbf{c_x}^{(i)}$. Then it calls the simulator $\mathcal{S}_{\sf Probe}({\sf csk}, {\sf C})$ and returns whatever the simulator returns.

\end{itemize}

The details of these oracles are given in Figure \ref{fig:sim_game_oracle}.

\begin{figure}[h]
	\begin{center}

		\begin{subfigure}[t]{0.49\textwidth}
		\begin{tabular}{l c}
			$\tilde{\mathcal{O}}_{\sf Enroll}({\sf csk}, {\sf esk}, {\sf pp}, \mathbf{x}^{(j)})$\\

			\hline

			$\mathbf{c_x}^{(j)} \gets {\sf Enroll}({\sf esk}, {\sf pp}, \mathbf{x}^{(j)})$ \\

			${\sf C} \gets {\sf C} \cup \{ {\sf Compare}({\sf csk}, {\sf pp}, \mathbf{c_x}^{(j)}, \mathbf{c_y}^{(i)}) \}_{i}$ \\

			$\mathbf{c_x}' \gets \mathcal{S}_{\sf Enroll}({\sf csk}, {\sf C} )$ \\
			
			\textbf{return} $\mathbf{c_x}'$
			
		\end{tabular}
		\end{subfigure}
		\begin{subfigure}[t]{0.49\textwidth}
		\begin{tabular}{l c}
			$\tilde{\mathcal{O}}_{\sf Probe}({\sf csk}, {\sf psk}, {\sf pp}, \mathbf{y}^{(j)})$\\

			\hline

			$\mathbf{c_y}^{(j)} \gets {\sf Probe}( {\sf psk}, {\sf pp}, \mathbf{y}^{(j)} ) $ \\

			${\sf C} \gets {\sf C} \cup \{ {\sf Compare}({\sf csk}, {\sf pp}, \mathbf{c_x}^{(i)}, \mathbf{c_y}^{(j)}) \}_{i}$ \\

			$\mathbf{c_y}' \gets \mathcal{S}_{\sf Probe}({\sf csk}, {\sf C} )$ \\

			\textbf{return} $\mathbf{c_y}'$
			
		\end{tabular}
		\end{subfigure}
	\end{center}
	\caption{Choice of the Oracle $\tilde{\mathcal{O}}$}
	\label{fig:sim_game_oracle}
\end{figure}


We define the advantage of an adversary $\mathcal{A}$ and a simulator $\mathcal{S} = (\mathcal{S}_0, \mathcal{S}_{\sf Enroll}, \mathcal{S}_{\sf Probe})$ in the simulation game of a scheme $\Pi$ associated with a family of distributions $\mathbb{B}$ as
\[
	\Adv^{\sf SIM}_{\Pi, \mathbb{B}, \mathcal{A}^{\mathcal{O}, \tilde{\mathcal{O}}}, \mathcal{S}} := |\Pr[{\sf SIM-Real}_{\Pi, \mathbb{B}}(\mathcal{A}^\mathcal{O}) \to 1] - \Pr[{\sf SIM-Ideal}_{\Pi, \mathbb{B}}(\mathcal{A}^{\tilde{\mathcal{O}}}, \mathcal{S}) \to 1]|.
\]

An authentication scheme $\Pi$ associated with a family of distributions $\mathbb{B}$ is called \emph{simulation secure} if for any PPT adversary $\mathcal{A}$, there exists a PPT simulator $\mathcal{S} = (\mathcal{S}_0, \mathcal{S}_{\sf Enroll}, \mathcal{S}_{\sf Probe})$ such that
\[
	\Adv^{\sf SIM}_{\Pi, \mathbb{B}, \mathcal{A}^{\mathcal{O}, \tilde{\mathcal{O}}}, \mathcal{S}} = \negl.
\]

%-------------------

\subsection*{Identification Game}
\label{sec:id_game}

In the identification game, we model the ability of an adversary who has access to the server's database of registered enrollments and tries to identify the user. The adversary $\mathcal{A}$ is given an enrollment message $\mathbf{c_x}^{(b)}$ and two distinct distributions $\mathcal{B}^{(0)}, \mathcal{B}^{(1)}$ that can be efficiently sampled. She tries to guess from which $\mathbf{c_x}^{(b)}$ is generated. The whole game is defined in Figure \ref{fig:id_game}.

\begin{figure}[h]
	\begin{center}
	\begin{tabular}{l c}
		${\sf Id}_{\Pi, \mathbb{B}}(\mathcal{A}^\mathcal{O})$\\

			\hline

			$b \getsdollar \{0, 1\}$ \\

			$\mathcal{B}^{(0)}, \mathcal{B}^{(1)} \getsdollar \mathbb{B}$ \\

			${\sf esk}, {\sf psk}, {\sf csk}, {\sf pp} \gets {\sf USetup}(1^\lambda)$ \\

			$ \mathbf{b}^{(b)} \getsdollar \mathcal{B}^{(b)}, \mathbf{x}^{(b)} \gets {\sf encodeEnroll}^{\mathcal{O}_{\sf aux}^{(b)}}(\mathbf{b}^{(b)})$ \\

			$\mathbf{c_x}^{(b)} \gets {\sf Enroll}({\sf esk}, {\sf pp}, \mathbf{x}^{(b)})$ \\

			$\tilde{b} \gets \mathcal{A}^{\mathcal{O}} ( {\sf pp}, \mathbf{c_x}^{(b)}, \mathcal{B}^{(0)}, \mathcal{B}^{(1)} )$ \\

			\textbf{return} $1_{\tilde{b} = b}$
			
	\end{tabular}
	\end{center}
	\caption{The Identification Game}
	\label{fig:id_game}
\end{figure}


The given oracle $\mathcal{O}$ can be any or more of the following three oracles:

\begin{itemize}
	\item $\mathcal{O}_{\sf Enroll}({\sf esk}, {\sf pp}, \cdot)$: On input $\mathbf{x}$, it outputs the enrollment message ${\sf Enroll}({\sf esk}, {\sf pp}, \mathbf{x})$.

	\item $\mathcal{O}_{\sf Probe}({\sf psk}, {\sf pp}, \cdot)$: On input $\mathbf{y}$, it outputs the probe message ${\sf Probe}({\sf psk}, {\sf pp}, \mathbf{y})$.

\end{itemize}

We define the advantage of an adversary $\mathcal{A}$ in the identification game of a scheme $\Pi$ associated with a family of distributions $\mathbb{B}$ as
\[
	\Adv^{\sf Id}_{\Pi, \mathbb{B}, \mathcal{A}^\mathcal{O}} := |\Pr[{\sf Id}_{\Pi}(\mathcal{A}^\mathcal{O}) \to 1] - \frac{1}{2}|.
\]

An authentication scheme $\Pi$ associated with a family of distributions $\mathbb{B}$ is called \emph{identification secure} if for any PPT adversary $\mathcal{A}$,
\[
	\Adv^{\sf Id}_{\Pi, \mathbb{B}, \mathcal{A}^\mathcal{O}} = \negl.
\]

%-------------------
%% Backup File

% \section*{Backup}
\subsection*{Privacy Game}

In the privacy game, we model the ability of the server who tries to learn the biometric template of a user. The adversary $\mathcal{A}$ is given the enrollment message $\mathbf{c_x}$, $t$ probe messages $\{ \mathbf{c_y}_i\}_{i=1}^t$ for some integer $t$, and oracle $\mathcal{O}$ and tries to find the template $\mathbf{b}$. The whole game is defined in Figure \ref{fig:privacy_game}.


\begin{figure}[h]
	\begin{center}
		\begin{tabular}{l c}
			${\sf Priv}_{\Pi}(\mathcal{A})$\\

			\hline

			${\sf csk}, {\sf pp_s} \gets {\sf SSetup(1^\lambda)}$ \\

			${\sf esk}, {\sf psk}, {\sf pp_u} \gets {\sf USetup}(1^\lambda, {\sf pp_s})$ \\

			${\sf pp}:= ({\sf pp_s}, {\sf pp_u})$ \\

			$ \mathbf{b} \getsdollar \mathcal{B}, \mathbf{x} \gets {\sf encodeEnroll}^{\mathcal{O}_{\sf aux}}(\mathbf{b})$ \\

			$\mathbf{c_x} \gets {\sf Enroll}({\sf esk}, {\sf pp}, \mathbf{x})$ \\

			$\{ \mathbf{b}_i \}_{i=1}^t \getsdollar \mathcal{B}^t$ \\ 

			$\{ \mathbf{c_y}_i \}_{i=1}^t \gets \left\{ {\sf Probe}( {\sf psk}, {\sf pp}, {\sf encodeProbe}^{\mathcal{O}_{\sf aux}}(\mathbf{b}_i) ) \right \}_{i=1}^t$ \\

			${\mathbf{\tilde{b}}} \gets \mathcal{A}^{\mathcal{O}} ( \sf{pp}, \mathbf{c_x}, \{ \mathbf{c_y}_i \}_{i=1}^t )$ \\

			\textbf{return} $1_{\mathbf{\tilde{b}} = \mathbf{b}}$
			
		\end{tabular}
	\end{center}
	\caption{The Privacy Game}
	\label{fig:privacy_game}
\end{figure}


The given oracle $\mathcal{O}$ can be any or more of the following three oracles:


\begin{itemize}

	\item $\mathcal{O}_{\sf Enroll}({\sf esk}, {\sf pp}, \cdot)$: On input $\mathbf{x}$, it outputs the enrollment message ${\sf Enroll}({\sf esk}, {\sf pp}, \mathbf{x})$.

	\item $\mathcal{O}_{\sf Probe}({\sf psk}, {\sf pp}, \cdot)$: On input $\mathbf{y}$, it outputs the probe message ${\sf Probe}({\sf psk}, {\sf pp}, \mathbf{y})$.

	\item $\mathcal{O}_{\sf Compare}({\sf csk}, {\sf pp}, \cdot, \cdot)$: On input $\mathbf{c_x}$ and $\mathbf{c_y}$, it outputs the comparison result ${\sf Compare}( {\sf csk}, {\sf pp}, \mathbf{c_x}, \mathbf{c_y} )$.

\end{itemize}

Note that if the enrollment secret key ${\sf esk}$, the probe secret key ${\sf psk}$, or the comparison secret key ${\sf csk}$ is an empty string in the scheme, then the corresponding oracles are naturally and implicitly given since the adversary can compute them herself.

We define the success probability of an adversary $\mathcal{A}$ in the privacy game of the scheme $\Pi$ as
\[
	\Adv^{\sf Priv}_{\Pi, \mathcal{A}^\mathcal{O}} := \Pr[{\sf Priv}_{\Pi}(\mathcal{A}^\mathcal{O}) \to 1].
\]

The authentication scheme $\Pi$ is called \emph{privacy} secure if for any PPT adversary $\mathcal{A}$,
\[
	\Adv^{\sf Priv}_{\Pi, \mathcal{A}^\mathcal{O}} = \negl.
\]

%-------------------

\subsection*{IND Game}
For empty comparison key ${\sf csk}$, to avoid trivial attacks, we need to restrict the adversary's queries.

\begin{definition}[Admissible Adversary]
\label{def:admissible}
	For an adversary $\mathcal{A}^\mathcal{O}$ in an IND game of a scheme $\Pi$ with empty comparison secret key, let $(\mathbf{x}_0^1, \mathbf{x}_1^1), (\mathbf{x}_0^2, \mathbf{x}_1^2), \cdots, (\mathbf{x}_0^{Q_{\sf Enroll}}, \mathbf{x}_1^{Q_{\sf Enroll}})$ be the queries to $\mathcal{O}_{\sf Enroll}^{\sf IND}$, and let $(\mathbf{y}_0^1, \mathbf{y}_1^1), (\mathbf{y}_0^2, \mathbf{y}_1^2),$ $\cdots, (\mathbf{y}_0^{Q_{\sf Probe}}, \mathbf{y}_1^{Q_{\sf Probe}})$ be the queries to $\mathcal{O}_{\sf Probe}^{\sf IND}$. We say the adversary $\mathcal{A}$ is \emph{admissible} if
	\[
		\begin{aligned}
			& {\sf Compare}( {\sf pp}, {\sf Enroll}({\sf esk}, {\sf pp}, \mathbf{x}_0^i), {\sf Probe}({\sf psk}, {\sf pp}, \mathbf{y}_0^j) ) \\
			= \; & {\sf Compare}( {\sf pp}, {\sf Enroll}({\sf esk}, {\sf pp}, \mathbf{x}_1^i), {\sf Probe}({\sf psk}, {\sf pp}, \mathbf{y}_1^j) )
		\end{aligned}
		\quad \forall i \in [Q_{\sf Enroll}], \forall j \in [Q_{\sf Probe}]
	\]
	

\end{definition}

This prevents the adversary's win from simply querying unmatched enrollment and probe message pairs and finding the difference from the public $\sf Compare$ function.

We define the success probability of an adversary $\mathcal{A}$ in the IND game of the scheme $\Pi$ as
\[
	\Adv^{\sf IND}_{\Pi, \mathcal{A}^\mathcal{O}} := \Pr[{\sf IND}_{\Pi}(\mathcal{A}^\mathcal{O}) \to 1].
\]

The authentication scheme $\Pi$ is called \emph{IND} secure if for all PPT adversaries $\mathcal{A}$ when $\Pi$ has a non-trivial comparison secret key, or for all admissible PPT adversaries $\mathcal{A}$ when $\Pi$ does not have a non-trivial comparison secret key,
\[
	\Adv^{\sf IND}_{\Pi, \mathcal{A^\mathcal{O}}} = \negl.
\]



%-------------------
%% Reference List
\nocite{*}
\printbibliography


\end{document}
