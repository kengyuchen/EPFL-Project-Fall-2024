\begin{frame}{Biometric Authentication}

\begin{itemize}
	\item An \emph{error-tolerant} approach to user verification.
	\begin{itemize}
		\item A server verifies identities by comparing the \emph{similarity}, instead of equivalence.
	\end{itemize}
	\pause

	\item Unlike password, biometrics reveal personal information and cannot be changed.
	\pause

	\item Possibly non-negligible false positive/negative rates.
\end{itemize}

\end{frame}


\begin{frame}{Biometric Authentication}

In this project, we formalize a biometric authenticaion scheme.
\pause

\begin{itemize}

	\item<2-> We formally define a biometric authenticaion scheme.

	\item<3-> We add a cryptographic layer on top of the authentication scheme to preserve privacy.

	\item<4-> We model two security notions of interest: \emph{unforgeability} and \emph{indistinguishability}
	\begin{itemize}
		\item<5-> Unforgeability: An adversary cannot \emph{impersonate} the user.
		\item<6-> Indistinguishability: The server cannot \emph{recognize} the user.
	\end{itemize}

	\item<7-> We analyze two instantiations from previous works.
	\begin{itemize}
		\item<8-> Function-hiding inner product functional encryption \cite{cryptoeprint:2023/481}.
		\item<9-> {\color<10>{trans}{Relational hash}} \cite{cryptoeprint:2014/394}.
	\end{itemize}

\end{itemize}


\end{frame}


\begin{frame}{Notation}

\begin{itemize}
	\item $\lambda$: the security parameter.
	
	\item $\poly$ ($\negl$) denotes a polynomial (negligible) function of $\lambda$.

	\item Sample a value $r$ from a distribution $\mathcal{D}$ (uniformly from a set $S$) is $r \getsdollar \mathcal{D}$ ($r \getsdollar S$).

\end{itemize}
    
\end{frame}
