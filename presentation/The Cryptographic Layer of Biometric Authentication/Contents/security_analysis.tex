
\begin{frame}{Security Analysis}

In this section, we will discuss

\begin{itemize}

	\item Function-hiding inner product functional encryption (fh-IPFE) \cite{cryptoeprint:2016/440}.

	\item An instantiation $\Pi$ using an fh-IPFE \cite{cryptoeprint:2023/481}.

	\item UF and IND security of $\Pi$.

\end{itemize}

In our project, we also discuss an instantiation using \emph{relational hash}.

\end{frame}


\subsection{Instantiation using fh-IPFE}


\begin{frame}{fh-IPFE}

\begin{definition}{Function-Hiding Inner Product Functional Encryption (adapted from \cite{cryptoeprint:2016/440})}
	A \emph{function-hiding inner product functional encryption} (fh-IPFE) scheme $\sf FE$ for a field $\mathbb{F}$ is composed of PPT algorithms:
	\begin{itemize}
	
		\item<2-> $\textsf{FE.Setup}(1^\lambda)$: Output the public parameter $\textsf{pp}$ and the master secret key $\textsf{msk}$.
	
		\item<3-> $\textsf{FE.KeyGen}(\textsf{msk}, \textsf{pp}, \mathbf{x})$: On input a vector $\mathbf{x} \in \mathbb{F}^k$, output the decryption key $f_\mathbf{x}$.  
	
		\item<4-> $\textsf{FE.Enc}(\textsf{msk}, \textsf{pp}, \mathbf{y})$: On input a vector $\mathbf{y} \in \mathbb{F}^k$, output the ciphertext $\mathbf{c_y}$. 
	
		\item<5-> $\textsf{FE.Dec}(\textsf{pp}, f_\mathbf{x}, \mathbf{c_y})$: Output a value $z \in \mathbb{F}$ or an error symbol $\bot$.
	
	\end{itemize}
	

\end{definition}

\end{frame}


\begin{frame}{fh-IPFE}

\begin{block}{Correctness}

	\textsf{FE} is \emph{correct} if $\forall (\textsf{msk}, \textsf{pp}) \gets \textsf{FE.Setup}(1^\lambda)$ and $ \mathbf{x}, \mathbf{y} \in \mathbb{F}^k$, we have
	\[
		\textsf{FE.Dec}( \textsf{pp}, \textsf{FE.KeyGen}(\textsf{msk}, \textsf{pp}, \mathbf{x}), \textsf{FE.Enc}(\textsf{msk}, \textsf{pp}, \mathbf{y}) ) = \mathbf{x} \mathbf{y}^T \in \mathbb{F}.
	\]

\end{block}

\end{frame}



\begin{frame}{Instantiation using fh-IPFE \cite{cryptoeprint:2023/481}}

\begin{itemize}
	\item<1-> $\textsf{getEnroll}^{\mathcal{O}_{\mathcal{B}}}()$, $\textsf{getProbe}^{\mathcal{O}_{\mathcal{B}}}()$: Output vectors in $ \{0, 1, \cdots, m \}^k$ for all $\mathcal{B} \in \mathbb{B}$.

	\item<2-> $\textsf{BioCompare}(\mathbf{b}, \mathbf{b}^\prime) \to \| \mathbf{b} - \mathbf{b}^\prime\|^2$.

	\item<3-> For a pre-defined real number $\tau \geq 0$,
	\[
		\textsf{Verify}(s) \to 
		\begin{cases} 
			1 & \text{if } \sqrt{s} \leq \tau \\
			0 & \text{if } \sqrt{s} > \tau 
		\end{cases}.
	\]

	\item<4-> An fh-IPFE $\textsf{FE}$ associated with a field $\mathbb{F} = \mathbb{Z}_q$.
	\begin{itemize}
		\item $q$ is a prime number larger than the maximum possible Euclidean distance $m^2 \cdot k$.
	\end{itemize}
\end{itemize}

\end{frame}


\begin{frame}{Instantiation using fh-IPFE \cite{cryptoeprint:2023/481}}

\begin{itemize}

	\item<1-> $\textsf{Setup}(1^\lambda)$: Run $\textsf{FE.Setup}(1^\lambda) \to \textsf{msk}, \textsf{pp}$ and output 
	\begin{itemize}
		\item $\textsf{esk} \gets (\textsf{msk}, \textsf{pp})$
		\item $\textsf{psk} \gets (\textsf{msk}, \textsf{pp})$
		\item $\textsf{csk} \gets \textsf{pp}$
	\end{itemize}

	\item<2-> $\textsf{Enroll}(\textsf{esk}, \mathbf{b})$: On input a template vector $\mathbf{b} = (b_1, b_2, \cdots, b_k)$,
	\begin{itemize}
		\item Encode $\mathbf{b}$ as $\mathbf{x} = (x_1, x_2, \cdots, x_{k+2}) = (b_1, b_2, \cdots, b_k, 1, \|\mathbf{b}\|^2)$.
		\item Run and output $\mathbf{c_x} \gets \textsf{FE.KeyGen}(\textsf{msk}, \textsf{pp}, \mathbf{x})$.
	\end{itemize}

	\item<3-> $\textsf{Probe}(\textsf{psk}, \mathbf{b}^\prime)$: On input a template vector $\mathbf{b}^\prime = (b_1^\prime, b_2^\prime, \cdots, b_k^\prime)$,
	\begin{itemize}
		\item Encode $\mathbf{b}^\prime$ as $\mathbf{y} = (y_1, y_2, \cdots, y_{k+2}) = (-2b_1^\prime, -2b_2^\prime, \cdots, -2b_k^\prime, \|\mathbf{b}^\prime\|^2, 1)$.
		\item Run and output $\mathbf{c_y} \gets \textsf{FE.Enc}(\textsf{msk}, \textsf{pp}, \mathbf{y})$.
	\end{itemize}

	\item<4-> $\textsf{Compare}(\textsf{csk}, \mathbf{c_x}, \mathbf{c_y)}$: Run and output $s \gets \textsf{FE.Dec}(\textsf{pp}, \mathbf{c_x}, \mathbf{c_y})$.

\end{itemize}

\end{frame}


\begin{frame}{Instantiation using fh-IPFE \cite{cryptoeprint:2023/481}}

By the correctness of $\sf FE$,
\[
	s = \textsf{FE.Dec}(\textsf{pp}, \mathbf{c_x}, \mathbf{c_y}) =  \mathbf{x} \mathbf{y}^T = \sum_{i=1}^k -2b_ib_i^\prime + \|\mathbf{b}\|^2 + \|\mathbf{b}^\prime\|^2 = \| \mathbf{b} - \mathbf{b}^\prime \|^2.
\]
which is equal to $\textsf{BioCompare}(\mathbf{b}, \mathbf{b}^\prime)$.
\pause

Note that in this instantiation,
\pause

\begin{itemize}
	\item $\textsf{esk} = \textsf{psk}$.
	\pause

	\item $\textsf{csk} = \textsf{pp}$, the public parameter of \textsf{FE}. Therefore, we offer $\textsf{csk}$ for all adversaries.

\end{itemize}


\end{frame}


\subsection{Security of fh-IPFE}


\begin{frame}{fh-IND of fh-IPFE}

Given an fh-IPFE scheme \textsf{FE}, define the \textsf{fh-IND} game \cite{cryptoeprint:2016/440}.

\centerline{
	\begin{minipage}{.4\textwidth}
	\begin{block}{ $\textsf{fh-IND}_{\textsf{FE}}(\mathcal{A})$ }
	\vskip -10pt
	\begin{algorithm}[H]
	\label{alg:ind-fh-IPFE}
	\begin{algorithmic}[1]
		\STATE $b \getsdollar \{0, 1\}$

		\STATE $\textsf{msk}, \textsf{pp} \gets \textsf{FE.Setup}(1^\lambda)$

		\STATE $\tilde{b} \gets \mathcal{A}^{\mathcal{O}_{\textsf{KeyGen}}, \mathcal{O}_{\textsf{Enc}}} ( \textsf{pp} )$

		\RETURN $1_{\tilde{b} = b}$
	\end{algorithmic}
	\end{algorithm}

	\end{block}
	\end{minipage}
}
\medskip

\begin{itemize}

	\item $\mathcal{O}_{\textsf{KeyGen}}(\cdot, \cdot)$: On input pair $(\mathbf{x}^{(0)}, \mathbf{x}^{(1)})$, output $\textsf{FE.KeyGen}(\textsf{msk}, \textsf{pp}, \mathbf{x}^{(b)} )$.

	\item $\mathcal{O}_{\textsf{Enc}}(\cdot, \cdot)$: On input pair $(\mathbf{y}^{(0)}, \mathbf{y}^{(1)})$, output $\textsf{FE.Enc}(\textsf{msk}, \textsf{pp}, \mathbf{y}^{(b)} )$.

\end{itemize}
\pause

A trivial adversary can ask $\mathcal{O}_{\textsf{KeyGen}}(\mathbf{x}^{(0)}, \mathbf{x}^{(1)})$ and $\mathcal{O}_{\textsf{Enc}}(\mathbf{y}^{(0)}, \mathbf{y}^{(1)})$ such that 
\[
	\mathbf{x}^{(0)} {\mathbf{y}^{(0)}}^T \neq \mathbf{x}^{(1)} {\mathbf{y}^{(1)}}^T.
\]

\end{frame}


\begin{frame}{fh-IND of fh-IPFE}

\begin{definition}{Admissible Adversary}

	Let $\mathcal{A}$ be an adversary in an \textsf{fh-IND} game, and let $ (\mathbf{x}_1^{(0)}, \mathbf{x}_1^{(1)}), \cdots, (\mathbf{x}_{Q_K}^{(0)}, \mathbf{x}_{Q_K}^{(1)})$ be its queries to $\mathcal{O}_{\textsf{KeyGen}}$ and $(\mathbf{y}_1^{(0)}, \mathbf{y}_1^{(1)}), \cdots, (\mathbf{y}_{Q_E}^{(0)}, \mathbf{y}_{Q_E}^{(1)})$ be its queries to $\mathcal{O}_{\textsf{Enc}}$.\pause \\
	\medskip
	We say $\mathcal{A}$ is \emph{admissible} if $\forall i \in [Q_K], \forall j \in [Q_E]$,
\[
	{\mathbf{x}^{(0)}_{i}} {\mathbf{y}^{(0)}_{j}}^T = {\mathbf{x}^{(1)}_{i}} {\mathbf{y}^{(1)}_{j}}^T
\]

\end{definition}
\pause

\begin{definition}{fh-IND Security}

	\textsf{FE} is called \emph{fh-IND} secure if for any admissible adversary $\mathcal{A}$, 
	\[
		\Adv_{\textsf{FE}, \mathcal{A}}^{\textsf{fh-IND}} := \left| \Pr[\textsf{fh-IND}_{\textsf{FE}}(\mathcal{A}) \to 1 ] - \frac{1}{2} \right| = \negl.
	\]

\end{definition}

\end{frame}


\begin{frame}{fh-IND Security}

\begin{itemize}
	
	\item<1-> fh-IND security is the standard notion of an fh-IPFE scheme.
	\begin{itemize}
		\item Constructions in \cite{cryptoeprint:2015/1255, 10.1007/978-3-319-45871-7_24, cryptoeprint:2016/440} are proven fh-IND.
	\end{itemize}

	\item<2-> However, fh-IND security is not sufficient for the UF security of $\Pi$.
	\begin{itemize}
		\item We found that instantiation $\Pi$ using \cite{cryptoeprint:2016/440} is not \textsf{option}-unforgeable for any \textsf{option}.
	\end{itemize}

	\item<3-> For this, we define another extra security notion of \textsf{FE}: \emph{RUF Security}.

\end{itemize}

\end{frame}


\begin{frame}{RUF of fh-IPFE}

Let $\gamma \geq 0$ be a real number and $\mathbb{F} = \Z_q$ for a prime number $q$. Define the \textsf{RUF} game.

\centerline{
	\begin{minipage}{.4\textwidth}
	\begin{block}{ $\textsf{RUF}^{\mathcal{O}, \gamma}_{\textsf{FE}}(\mathcal{A})$ }
	\vskip -10pt
	\begin{algorithm}[H]
	\label{alg:ruf-fh-IPFE}
	\begin{algorithmic}[1]
		\STATE $\mathbf{r} \getsdollar \mathbb{F}^{k}$ \label{alg:oracle-ruf-fh-IPFE:r}

		\STATE $\textsf{msk}, \textsf{pp} \gets \textsf{FE.Setup}(1^\lambda)$

		\STATE $\mathbf{c} \gets \textsf{FE.KeyGen}(\textsf{msk}, \textsf{pp}, \mathbf{r})$

		\STATE $\mathbf{\tilde{z}} \gets \mathcal{A}^{\mathcal{O}} ( \textsf{pp}, \mathbf{c} )$

		\STATE $s \gets \textsf{FE.Dec}(\textsf{pp}, \mathbf{c}, \mathbf{\tilde{z}} )$

		\RETURN $1_{s \leq \gamma}$
	\end{algorithmic}
	\end{algorithm}

	\end{block}
	\end{minipage}
}
\medskip

Oracle $\mathcal{O}$ can be nothing or include

\begin{itemize}

	\item $\mathcal{O}^\prime_{\textsf{KeyGen}}(\cdot)$: On input $\mathbf{x}^\prime$, output $\textsf{FE.KeyGen}(\textsf{msk}, \textsf{pp}, \mathbf{x}^\prime)$.
	
	\item $\mathcal{O}^\prime_{\textsf{Enc}}(\cdot)$: On input $\mathbf{y}^\prime$, output $\textsf{FE.Enc}(\textsf{msk}, \textsf{pp}, \mathbf{y}^\prime)$. 
\end{itemize}


\end{frame}


\begin{frame}{RUF of fh-IPFE}

\begin{itemize}
	\item We forbid the adversary to return $\mathbf{\tilde{z}}$ that is an answer of $\mathcal{O}^\prime_{\textsf{Enc}}(\cdot)$.
	\begin{itemize}
		\item Otherwise, returning $\mathcal{O}^\prime_{\textsf{Enc}}(\mathbf{0})$ wins with probability $1$.
	\end{itemize}

\end{itemize}

\begin{definition}{RUF Security}

	\textsf{FE} is called $\mathcal{O}$-RUF secure for a real number $\gamma$ if for any adversary $\mathcal{A}$,
	\[
		\Adv_{\textsf{FE}, \mathcal{A}}^{\textsf{RUF}, \mathcal{O}, \gamma} := \Pr[\textsf{RUF}^{\mathcal{O}, \gamma}_{\textsf{FE}}(\mathcal{A}) \to 1 ] = \negl.
	\]

\noindent We say $\textsf{FE}$ is RUF secure if it is $\{ \mathcal{O}^\prime_{\textsf{KeyGen}}, \mathcal{O}^\prime_{\textsf{Enc}} \}$-RUF secure.

\end{definition}

\end{frame}


\begin{frame}{RUF of fh-IPFE}

\begin{itemize}

	\item<1-> RUF security is a new notion of an fh-IPFE scheme.

	\item<2-> In our project, we provide two theorems to obtain RUF security.

\end{itemize}

\visible<3->{
	\begin{theorem}
		If \textsf{FE} is fh-IND, and if the RUF adversary can only return $\mathbf{\tilde{z}}$ that is an encryption of a nonzero vector, then $\textsf{FE}$ is $\mathcal{O}^\prime_{\textsf{KeyGen}}$-RUF for any $\gamma \leq \|\mathbb{F}\|$.
	\end{theorem}
}

\visible<4->{
	\begin{theorem}
		Given an sEUF-CMA digital signature scheme \textsf{Sig} and any fh-IPFE $\textsf{FE}$, we can obtain an fh-IPFE $\textsf{FE}^\prime$ that is RUF for any $\gamma$.
	\end{theorem}
}

\visible<5->{We provide details in \hyperlink{sec:appendix-RUF}{Appendix - Achievability of RUF Security}.}

\end{frame}


\subsection{Security of Instantiation using fh-IPFE}


\begin{frame}{Security of Instantiation using fh-IPFE}

For the rest of this section, let $\Pi$ be a biomtric authentication scheme instantiated by an fh-IPFE \textsf{FE}.
\pause
In our project, we show
\pause

\begin{theorem}
	For any distribution family $\mathbb{B}$, if \textsf{FE} is fh-IND and $\mathcal{O}^\prime_{\textsf{KeyGen}}$-RUF for a $\gamma \geq \tau^2$, then $\Pi$ is $\{ \mathbf{c_x}, \textsf{csk}, \mathcal{O}_\mathcal{B}, \mathcal{O}_{\textsf{Enroll}} \}$-UF. 
\end{theorem}
\pause

\begin{theorem}
	For any distribution family $\mathbb{B}$, if \textsf{FE} is fh-IND and $\mathcal{O}^\prime_{\textsf{Enc}}$-RUF for a $\gamma \geq \tau^2$, then $\Pi$ is $\{\mathbf{c_x}, \textsf{csk}, \mathcal{O}_\mathcal{B}, \mathcal{O}_{\textsf{Probe}} \}$-UF. 
\end{theorem}
\pause

\begin{theorem}
	For any distribution family $\mathbb{B}$ satisfying some "reasonable conditions", if \textsf{FE} is fh-IND, then $\Pi$ is IND.

\end{theorem}

\end{frame}


\begin{frame}{Security of Instantiation using fh-IPFE}

{\color{trans}{
For the rest of this section, let $\Pi$ be a biomtric authentication scheme instantiated by an fh-IPFE \textsf{FE}.
In our project, we show
}}

\begin{theorem}
	For any distribution family $\mathbb{B}$, if \textsf{FE} is fh-IND and $\mathcal{O}^\prime_{\textsf{KeyGen}}$-RUF for a $\gamma \geq \tau^2$, then $\Pi$ is $\{ \mathbf{c_x}, \textsf{csk}, \mathcal{O}_\mathcal{B}, \mathcal{O}_{\textsf{Enroll}} \}$-UF. 
\end{theorem}


\begin{theorem}
{\color{trans}{
	For any distribution family $\mathbb{B}$, if \textsf{FE} is fh-IND and $\mathcal{O}^\prime_{\textsf{Enc}}$-RUF for a $\gamma \geq \tau^2$, then $\Pi$ is $\{\mathbf{c_x}, \textsf{csk}, \mathcal{O}_\mathcal{B}, \mathcal{O}_{\textsf{Probe}} \}$-UF.
}}
\end{theorem}


\begin{theorem}
{\color{trans}{
	For any distribution family $\mathbb{B}$ satisfying some "reasonable conditions", if \textsf{FE} is fh-IND, then $\Pi$ is IND.
}}
\end{theorem}

\end{frame}


\begin{frame}{$\{ \mathbf{c_x}, \textsf{csk}, \mathcal{O}_\mathcal{B}, \mathcal{O}_{\textsf{Enroll}} \}$-UF}

\begin{proof}[Proof Sketch]

Given an adversary $\mathcal{A}$ in the $\textsf{UF}_{\Pi, \mathbb{B}, \textsf{option}}$ game, we build a reduction adversary $\mathcal{R}$ in the fh-IND game such that:

\begin{itemize}
	\item<2-> If the challenge bit $b = 0$, $\mathcal{R}$ simulates a $\textsf{UF}_{\Pi, \mathbb{B}, \textsf{option}}(\mathcal{A})$ game.

	\item<3-> If the challenge bit $b = 1$, $\mathcal{R}$ simulates a $\textsf{RUF}^{\mathcal{O}^\prime_{\textsf{KeyGen}}, \gamma}_{\textsf{FE}}(\mathcal{A}^\prime)$ game, for some $\mathcal{A}^\prime$.

	\item<4-> Advantage of $\mathcal{R}$ is bounded by the difference between advantages of $\mathcal{A}$ and $\mathcal{A}^\prime$, and
	\[
		\Adv^{\textsf{UF}}_{\Pi, \mathbb{B}, \mathcal{A}, \textsf{option}} \leq 4 \cdot \Adv_{\textsf{FE}, \mathcal{R}}^{\textsf{fh-IND}} + \Adv_{\textsf{FE}, \mathcal{A}^\prime}^{\textsf{RUF},\mathcal{O}^\prime_{\textsf{KeyGen}}, \gamma} = \negl.
	\]

	\item<5-> $\mathcal{O}_{\textsf{Enroll}}$ is "encoding + $\textsf{FE.KeyGen}$". We can simulate $\mathcal{O}_{\textsf{Enroll}}$ by $\mathcal{O}_{\textsf{KeyGen}}$ in fh-IND game and $\mathcal{O}^\prime_{\textsf{KeyGen}}$ in $\textsf{RUF}$ game. 
	
	\item<6-> $\mathcal{R}$ never calls $\mathcal{O}_{\textsf{Enc}}$, so it is an admissible adversary.

\end{itemize}

\end{proof}

\end{frame}


\begin{frame}

\centerline{
	\begin{minipage}{.5\textwidth}
	\begin{block}{ $\mathcal{R}^{\mathcal{O}_{\textsf{KeyGen}}, \mathcal{O}_{\textsf{Enc}}}(\textsf{pp})$ }
	\vskip -8pt
	\begin{algorithm}[H]
	\label{alg:red:ind-uf-OB-Enroll}
	\begin{algorithmic}[1]
		\STATE $\mathcal{B} \getsdollar \mathbb{B}, \quad \mathbb{B} \gets \mathbb{B} \setminus \mathcal{B}$ \label{alg:red:ind-uf-OB-Enroll:B}

		\STATE $\mathbf{b} = (b_1, \cdots, b_k) \gets \textsf{getEnroll}^{\mathcal{O}_{\mathcal{B}}}()$

		\STATE $\mathbf{x} \gets (b_1, \cdots, b_k, 1, \|\mathbf{b}\|^2)$

		\STATE $\mathbf{r} \getsdollar \mathbb{F}^{k+2}$
		
		\STATE $\mathbf{c} \gets \mathcal{O}_{\textsf{KeyGen}}(\mathbf{x}, \mathbf{r})$ \label{alg:red:ind-uf-OB-Enroll:c}

		\STATE ${\mathbf{\tilde{z}}} \gets {\mathcal{A}}^{\mathcal{O}_\mathcal{B}, \mathcal{O}_{\textsf{Enroll}} } ( \mathbf{c}, \textsf{pp})$ \label{alg:red:ind-uf-OB-Enroll:A}

		\STATE $s \gets \textsf{FE.Dec}( \textsf{pp}, \mathbf{c}, \mathbf{\tilde{z}} )$

		\IF{$\textsf{Verify}(s) = 1$} \label{alg:red:ind-uf-OB-Enroll:verify}
			\RETURN $\tilde{b} = 0$
		\ELSE
			\RETURN $\tilde{b} \getsdollar \{0, 1\}$
		\ENDIF

	\end{algorithmic}
	\end{algorithm}

	\end{block}
	\end{minipage}
}
\medskip


\end{frame}



\begin{frame}{Security of Instantiation using fh-IPFE}

{\color{trans}{
For the rest of this section, let $\Pi$ be a biomtric authentication scheme instantiated by an fh-IPFE \textsf{FE}.
In our project, we show
}}

\begin{theorem}
{\color{trans}{
	For any distribution family $\mathbb{B}$, if \textsf{FE} is fh-IND and $\mathcal{O}^\prime_{\textsf{KeyGen}}$-RUF for a $\gamma \geq \tau^2$, then $\Pi$ is $\{ \mathbf{c_x}, \textsf{csk}, \mathcal{O}_\mathcal{B}, \mathcal{O}_{\textsf{Enroll}} \}$-UF.
}}
\end{theorem}


\begin{theorem}
	For any distribution family $\mathbb{B}$, if \textsf{FE} is fh-IND and $\mathcal{O}^\prime_{\textsf{Enc}}$-RUF for a $\gamma \geq \tau^2$, then $\Pi$ is $\{\mathbf{c_x}, \textsf{csk}, \mathcal{O}_\mathcal{B}, \mathcal{O}_{\textsf{Probe}} \}$-UF.
\end{theorem}


\begin{theorem}
{\color{trans}{
	For any distribution family $\mathbb{B}$ satisfying some "reasonable conditions", if \textsf{FE} is fh-IND, then $\Pi$ is IND.
}}
\end{theorem}

\end{frame}


\begin{frame}{$\{ \mathbf{c_x}, \textsf{csk}, \mathcal{O}_\mathcal{B}, \mathcal{O}_{\textsf{Probe}} \}$-UF}

\begin{block}{Proof Sketch.}

Given an adversary $\mathcal{A}$ in the $\textsf{UF}_{\Pi, \mathbb{B}, \textsf{option}}$ game, we build a reduction adversary $\mathcal{R}$ in the fh-IND game such that:

\begin{itemize}
	\item<2-> If the challenge bit $b = 0$, $\mathcal{R}$ simulates a $\textsf{UF}_{\Pi, \mathbb{B}, \textsf{option}}(\mathcal{A})$ game.

	\item<2-> If the challenge bit $b = 1$, $\mathcal{R}$ simulates a $\textsf{RUF}^{\mathcal{O}^\prime_{\textsf{Enc}}, \gamma}_{\textsf{FE}}(\mathcal{A}^\prime)$ game, for some $\mathcal{A}^\prime$.

	\item<3-> To let $\mathcal{R}$ be admissible, we cannot directly simulate $\mathcal{O}_{\textsf{Probe}}$ by $\mathcal{O}_{\textsf{Enc}}$.
	\begin{itemize}
		\item $\mathcal{R}$ uses $\mathcal{O}_{\textsf{KeyGen}}(\mathbf{x}, \mathbf{r})$ to prepare $\mathbf{c}$ for $\mathcal{A}$ and $\mathcal{A}^\prime$.
	\end{itemize}

	\item<4-> We simulate $\mathcal{O}_{\textsf{Probe}}( \textsf{psk}, \mathbf{b}^\prime )$ by
	\begin{itemize}
		\item Encode $\mathbf{b}^\prime$ to $\mathbf{y}^\prime = (-2b_1^\prime, \cdots, \allowbreak -2b_k^\prime, \|\mathbf{b}^\prime\|^2, 1)$
		\item Compute $d \gets \mathbf{x}{\mathbf{y}^\prime}^T$ and find a vector $\mathbf{y}^{\prime\prime}$ such that $\mathbf{r}{\mathbf{y}^{\prime\prime}}^T = d$
		\item $\mathcal{O}_{\textsf{Enc}}(\mathbf{y}^\prime, {\mathbf{y}^{\prime\prime}})$.
	\end{itemize}

\end{itemize}
\end{block}

\end{frame}


\begin{frame}{}

\begin{proof}[]

\begin{itemize}
	
	\item<1-> $\mathcal{R}$ is now admissible, but then we have to simulate the tweaked $\mathcal{O}_{\textsf{Probe}}$ in $\textsf{RUF}^{\mathcal{O}^\prime_{\textsf{Enc}}, \gamma}_{\textsf{FE}}(\mathcal{A}^\prime)$ game.

	\item<2-> Advantage of $\mathcal{R}$ is bounded by the difference between advantages of $\mathcal{A}$ and $\mathcal{A}^\prime$, and
	\[
		\Adv^{\textsf{UF}}_{\Pi, \mathbb{B}, \mathcal{A}, \textsf{option}} \leq 4 \cdot \Adv_{\textsf{FE}, \mathcal{R}}^{\textsf{fh-IND}} + \Adv_{\textsf{FE}, \mathcal{A}^\prime}^{\textsf{RUF}, \mathcal{O}^\prime_{\textsf{Enc}}, \gamma} + \frac{k+2}{q^{k+2}-1} + \frac{1}{q^{k+2}} = \negl.
	\]
\end{itemize}
\end{proof}

\end{frame}

\begin{frame}

\centerline{
	\begin{minipage}{.5\textwidth}
	\begin{block}{ $\mathcal{R}^{\mathcal{O}_{\textsf{KeyGen}}, \mathcal{O}_{\textsf{Enc}}}(\textsf{pp})$ }
	\vskip -8pt
	\begin{algorithm}[H]
	\label{alg:red:ind-uf-OB-Probe}
	\begin{algorithmic}[1]
		\STATE $\mathcal{B} \getsdollar \mathbb{B}, \quad \mathbb{B} \gets \mathbb{B} \setminus \mathcal{B}$ \label{alg:red:ind-uf-OB-Probe:B}

		\STATE $\mathbf{b} = (b_1, \cdots, b_k) \gets \textsf{getEnroll}^{\mathcal{O}_{\mathcal{B}}}()$

		\STATE $\mathbf{x} \gets (b_1, \cdots, b_k, 1, \|\mathbf{b}\|^2)$

		\STATE $\mathbf{r} \getsdollar \mathbb{F}^{k+2}$

		\STATE $\mathbf{c} \gets \mathcal{O}_{\textsf{KeyGen}}(\mathbf{x}, \mathbf{r})$

		\STATE ${\mathbf{\tilde{z}}} \gets \mathcal{A}^{\mathcal{O}_{\mathcal{B}}, \mathcal{O}_{\textsf{Probe}} } ( \mathbf{c}, \textsf{pp})$

		\IF{$\mathbf{\tilde{z}}$ is equal to any output of $\mathcal{O}_{\textsf{Probe}}$}

			\RETURN $\bot$

		\ENDIF

		\STATE $s \gets \textsf{FE.Dec}( \textsf{pp}, \mathbf{c}, \mathbf{\tilde{z}} )$

		\IF{$\textsf{Verify}(s) = 1$} \label{alg:red:ind-uf-OB-Probe:verify}
			\RETURN $\tilde{b} = 0$
		\ELSE
			\RETURN $\tilde{b} \getsdollar \{0, 1\}$
		\ENDIF

	\end{algorithmic}
	\end{algorithm}

	\end{block}
	\end{minipage}
}
\medskip


\end{frame}


\begin{frame}{Security of Instantiation using fh-IPFE}

{\color{trans}{
For the rest of this section, let $\Pi$ be a biomtric authentication scheme instantiated by an fh-IPFE \textsf{FE}.
In our project, we show
}}

\begin{theorem}
{\color{trans}{
	For any distribution family $\mathbb{B}$, if \textsf{FE} is fh-IND and $\mathcal{O}^\prime_{\textsf{KeyGen}}$-RUF for a $\gamma \geq \tau^2$, then $\Pi$ is $\{ \mathbf{c_x}, \textsf{csk}, \mathcal{O}_\mathcal{B}, \mathcal{O}_{\textsf{Enroll}} \}$-UF.
}}
\end{theorem}


\begin{theorem}
{\color{trans}{
	For any distribution family $\mathbb{B}$, if \textsf{FE} is fh-IND and $\mathcal{O}^\prime_{\textsf{Enc}}$-RUF for a $\gamma \geq \tau^2$, then $\Pi$ is $\{\mathbf{c_x}, \textsf{csk}, \mathcal{O}_\mathcal{B}, \mathcal{O}_{\textsf{Probe}} \}$-UF.
}}
\end{theorem}


\begin{theorem}
	For any distribution family $\mathbb{B}$ satisfying some "reasonable conditions", if \textsf{FE} is fh-IND, then $\Pi$ is IND.
\end{theorem}

\end{frame}


\begin{frame}{IND}

\begin{block}{Assumption 1}
Let $t$ be an integer. Assume that for any distribution $\mathcal{B} \in \mathbb{B}$, the following distribution is identical.
\[
	\mathcal{D}_\mathcal{B}(t) := \left( \textsf{BioCompare}(\mathbf{b}, \mathbf{b}^{(1)}), \textsf{BioCompare}(\mathbf{b}, \mathbf{b}^{(2)}), \cdots, \textsf{BioCompare}(\mathbf{b}, \mathbf{b}^{(t)}) \right)
\]
where $\mathbf{b} \gets \textsf{getEnroll}^{\mathcal{O}_\mathcal{B}}()$ and $ \mathbf{b}^{(i)} \gets \textsf{getProbe}^{\mathcal{O}_\mathcal{B}}()$ for all $i \in [t]$.

\end{block}
\pause

Note that indistinguishability of $\mathcal{D}_\mathcal{B}(t)$ for $\mathcal{B} \in \mathbb{B}$ is a necessary condition to achieve IND security because
\[
	\left( \textsf{Compare}(\textsf{csk}, \mathbf{c_x}, \mathbf{c_y}^{(1)}), \cdots, \textsf{Compare}(\textsf{csk}, \mathbf{c_x}, \mathbf{c_y}^{(t)}) \right) = \mathcal{D}_\mathcal{B}(t)
\]
where $b$ is the challenge bit.


\end{frame}


\begin{frame}{IND}

\begin{theorem}
	For any distribution family $\mathbb{B}$ satisfying Assumption 1 and having a true positive rate $\textsf{TP} > \frac{1}{\poly}$, if \textsf{FE} is fh-IND, then $\Pi$ is IND.
\end{theorem}
\pause

\begin{proof}[Proof Sketch.]

Given an adversary $\mathcal{A}$ in the $\textsf{IND}_{\Pi, \mathbb{B}}$ game, we build a reduction adversary $\mathcal{R}$ in the fh-IND game such that:
\pause

\begin{itemize}
	\item $\mathcal{R}$ first samples $\mathcal{B}^{(0)}$ and $\mathcal{B}^{(1)}$.
	\pause

	\item $\mathcal{R}$ then calls $\mathbf{c_x} \gets \mathcal{O}_{\textsf{KeyGen}}(\mathbf{x}^{(0)}, \mathbf{x}^{(1)})$, where $\mathbf{x}^{(0)}$ and $\mathbf{x}^{(1)}$ are created from $\mathcal{B}^{(0)}$ and $\mathcal{B}^{(1)}$.
	\pause

	\item $\mathcal{R}$ prepares $\mathbf{y}^{(0)}$ and $ \mathbf{y}^{(1)}$ from $\mathcal{B}^{(0)}$ and $\mathcal{B}^{(1)}$ in a way that, $\mathbf{x}^{(0)}{\mathbf{y}^{(0)}}^T = \mathbf{x}^{(1)}{\mathbf{y}^{(1)}}^T$, and calls $\mathbf{c_y} \gets \mathcal{O}_{\textsf{Enc}}(\mathbf{y}^{(0)}, \mathbf{y}^{(1)} )$.
	\pause

	\item By Assumption 1, $\mathbf{y}^{(0)}$ and $ \mathbf{y}^{(1)}$ still follow the correct distribution.
\end{itemize}

\end{proof}

\end{frame}


\begin{frame}


\centerline{
	\begin{minipage}{.7\textwidth}
	\begin{block}{ $\mathcal{R}^{\mathcal{O}_{\textsf{KeyGen}}, \mathcal{O}_{\textsf{Enc}}}(\textsf{pp})$ }
	\vskip -8pt
	
	\begin{algorithm}[H]
	\small
	\label{alg:red:ind-ind}
	\begin{algorithmic}[1]
		\STATE $\mathcal{B}^{(0)} \getsdollar \mathbb{B}, \quad \mathbb{B} \gets \mathbb{B} \setminus \mathcal{B}^{(0)}$ \label{alg:red:ind-ind:B0}
		
		\STATE $\mathcal{B}^{(1)} \getsdollar \mathbb{B}, \quad \mathbb{B} \gets \mathbb{B} \setminus \mathcal{B}^{(1)}$ \label{alg:red:ind-ind:B1}

		\STATE $\mathbf{b}^{(0)} \gets \textsf{getEnroll}^{\mathcal{O}_{\mathcal{B}^{(0)}}}(), \mathbf{x}^{(0)} \gets (b_1^{(0)}, \cdots, b_k^{(0)}, 1, \|\mathbf{b}^{(0)}\|^2)$
		
		\STATE $\mathbf{b}^{(1)} \gets \textsf{getEnroll}^{\mathcal{O}_{\mathcal{B}^{(1)}}}(), \mathbf{x}^{(1)} \gets (b_1^{(1)}, \cdots, b_k^{(1)}, 1, \|\mathbf{b}^{(1)}\|^2)$
		
		\STATE $\mathbf{c_x} \gets \mathcal{O}_{\textsf{KeyGen}}(\mathbf{x}^{(0)}, \mathbf{x}^{(1)})$ \label{alg:red:ind-ind:cx}

		\FOR{$i = 1$ to $t$}
			
			\STATE ${\mathbf{b}^\prime}^{(0)} \gets \textsf{getProbe}^{\mathcal{O}_{\mathcal{B}^{(0)}}}()$

			\STATE $\mathbf{y}^{(0)} \gets (-2{b_1^\prime}^{(0)}, \cdots, -2{b_k^\prime}^{(0)}, \| {\mathbf{b}^\prime}^{(0)} \|^2, 1) $

			\REPEAT
				
				\STATE ${\mathbf{b}^\prime}^{(1)} \gets \textsf{getProbe}^{\mathcal{O}_{\mathcal{B}^{(1)}}}()$

				\STATE $\mathbf{y}^{(1)} \gets (-2{b_1^\prime}^{(1)}, \cdots, -2{b_k^\prime}^{(1)}, \| {\mathbf{b}^\prime}^{(1)} \|^2, 1)$

			\UNTIL{ $\mathbf{x}^{(0)} {\mathbf{y}^{(0)}}^T = \mathbf{x}^{(1)} {\mathbf{y}^{(1)}}^T$ } \label{alg:red:ind-ind:while}
			
			\STATE $\mathbf{c_y}^{(i)} \gets \mathcal{O}_{\textsf{Enc}}(\mathbf{y}^{(0)}, \mathbf{y}^{(1)})$ \label{alg:red:ind-ind:cy}

		\ENDFOR

		\STATE $\tilde{b} \gets {\mathcal{A}}^{\mathcal{O}_{\mathcal{B}^{(0)}}, \mathcal{O}_{\mathcal{B}^{(1)}} } (\textsf{pp}, \mathbf{c_x}, \{ \mathbf{c_y}^{(i)} \}_{i=1}^t )$ \label{alg:red:ind-ind:A}

		\RETURN $\tilde{b}$

	\end{algorithmic}
	\end{algorithm}

	\end{block}
	\end{minipage}
}
\medskip


\end{frame}
